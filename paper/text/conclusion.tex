\section{Conclusion}
\label{sec:conclusion}

Through our bird's eye view of \ML applied to \RE we have observed that in the
past couple of decades a good amount of research has been done on how to bring
these two worlds together. The stakes are high: while requirements engineering
is currently a domain under intensive research, attempts to address its
challenges academically have translated into few results in practice. Free-form
text-based tools with light-weight structuring capabilities such as DOORS are
now the norm in practice. Requirements elicitation, analysis, validation and
management keep on relying on human expertise and talent. While academics often insist that
better formalization brings advantages, the languages in which requirements are
formalized do not match the need that stakeholders in the \RE process
(technical and non-technical) need to communicate through artifacts that are
intelligible to all.

While not overstating the potential of \ML, which is has its own challenges to
overcome such as coarseness of the learned models, overfitting or hungriness
for data, we have provided in this article indications that \ML might become a cornerstone in \RE. For now, it
seems like the domain is now undergoing a pre-scientific phase: the studies we
have analyzed seldom compare themselves with the state-of-the-art (see
\tab{summary}). This suggests that the current body of research is composed of
new ideas, which have not yet been validated to its full extent by the
scientific or industrial communities. We thus call for a more extensive survey to validate
the preliminary conclusions we present in this work.
