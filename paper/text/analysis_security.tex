\subsubsection{Security Requirements} 

Due to the orthogonal character of their impact on a system, \emph{security}
requirements are notoriously difficult to identify, objectify and
quantify~\cite{Rashid:2016}. Also during requirement specification, it very
often happens that security requirements are masked by functional requirements (but can be
deduced from the context of the domain the system operates in)~\cite{Riaz:2014}.
Because of this, it often happens in practice that security requirements are
often only marginally tackled during system construction, paving the way
to potentially catastrophic consequences.
Machine learning can be of use here by aiding in the identification of segments
of text that describe security requirements. This is a \emph{classification}
problem: given a text, identify which parts of it correspond to which type of
security issues.

	Jindal\etal~\cite{Jildal:2016} automatically learn decision trees that can be
used to classify security requirements as \emph{authentication}, \emph{access
control}, \emph{encryption} or \emph{data integrity}. Preprocessing of the data
is done by stemming relevant terms and the \emph{features} used
are such terms.

	Riaz and her colleagues~\cite{Riaz:2014} use the k-nearest neighbors algoritm to
classify sentences in requirements documents as \emph{confidentiality},
\emph{integrity}, \emph{authentication}, \emph{availability},
\emph{accountability} or \emph{privacy} requirements. In order to find adequate
sentences and provide context to the classifier, the authors start by finding a
type for each sentence among the possibilities \emph{title}, \emph{list start},
\emph{list element} or \emph{normal sentence}. For the classification the
authors use the number of word transformations needed to go from one term in one sentence to a term in
another sentence. The classifier is trained using requirements sentences from
the healthcare domain that are manually classified. A particularity of the
approach is that each security requirement type is associated to a template that
helps in translating the security requirements into functional requirements in
order to ease during the implementation of the final system.

\levi{\cite{Knauss:2011}} 