\section{Introduction}

This is an introduction~\cite{Domingos:2015}
\noindent
\textbf{RQ1:} How is \ML used inside \RE tools?

\noindent
\textbf{RQ2:} Is using \ML inside \RE tools beneficial?

\noindent
\textbf{RQ3}: What \ML task(s) is (are) used for which purpose in \RE?

\noindent
\textbf{RQ4}: Which model types are used to perform the \RE tasks?

\noindent
\textbf{RQ5}: How are \ML features extracted/selected to guide \RE tasks?
\parisa{just kept the old RQ for comparision}
Machine learning algorithms are shown to have considerable practical importance in many application domains. Especially in the domains where there are large databases and a need for identifying some consistencies or, domains where the program, needs to adapt itself to changes~\cite{Zhang2002}. Requirements engineering is a critical part of software engineering and it seems appropriate to use machine learning methods for requirements engineering tasks while the requirements specification documents are mainly in natural language and, ML can enhance the performance at some tasks by replicating human processing. \\
This paper aims to present a survey of how ML is benefiting the existing RE approaches. More precisely, we pursue the following research questions: \\
\noindent
\textbf{RQ1:} What is the current state of the practice in \ML${\&}$\RE?

\noindent
\textbf{RQ2:} What types of learning methods are there available at our disposal?

\noindent
\textbf{RQ3:} Which \RE problems can tackle with \ML methods?

\noindent
\textbf{RQ4:} Is using \ML methods improving the \RE process?

To achieve our goals we performed a literature review, which was split into data preparation, data collection, and data analysis phase. First, a search string was prepared based on the research questions, then a search was performed over a predefined set of databases and all identified studies were assessed by means of title and abstract. Our literature review is not an exhaustive study of the field rather we are offering a snapshot of the current state-of-the-art by borrowing some techniques from Systematic Literature Review.\\
The major contributions of this article are as follows:
\begin{itemize}
    \item We provide an overview of the current state of the art of the challenges \RE faces that may be handled through \ML techniques. Thereby we focus on two important aspects:
    \begin{itemize}
        \item Providing an overview of the \ML problem categories (classification, regression, clustering, etc.) in use for the support of \RE processes (elicitation, analysis, validation and, management).
        \item Providing an overview of the common \ML approaches (decision tree, K-Nearest Neighbors, Naive Bayesian, etc.) for tackling \RE problems and the data sets if available. 
    \end{itemize}
    \item We analyze the literature to discover trends and lessons on the use of \ML in \RE.
\end{itemize}
The paper is organized as follows. The rest of this section provides a background knowledge of \ML and \RE. Section \ref{sec:contributions} provides the overview of the issues that handled by \ML in different \RE process steps and, in Section \ref{sec:discussion}, we discuss the major study findings followed by a table for summarizing them. Finally, in Section \ref{sec:threats_to_validity} we state the possible issues that may affect the study validity.
\subsection{Machine Learning}

Machine-Learning (\ML) \cite{mitchell:1997} is a range of algorithm to
approximate functions and discover patterns in data.
Historically, models and heuristics are human-built exhaustive prescriptions of
how a system should behave. \ML is grounded on different premises:
rather than relying on humans to input all the possible cases the system can
handle, the field attempts to extrapolate patterns from a representative
set of examples that illustrate the expected behaviors. The way in which a
learning algorithm operates attempts to emulate the way in which humans learn:
from a set of examples, a general model for a behavior is induced.

Many learning algorithms exist, based on different visions of how learning
happens in practice~\cite{Domingos:2015}. All these algorithms have in common
the notion of \emph{features}. Features correspond to characteristics of what is
being learned and provide the grounds for the algorithm to abstract from the
complexities of the real world. Assume for example that an algorithm should
learn, based on a brain scan of a medical patient, to decide whether that
patient has brain cancer or not. A number of \emph{features} such as for example
the ``number of irregular objects in the scan'', the ``color of such objects'',
the ``disposition of such objects'' would be provided to the algorithm.
Additionally, the algorithm is fed a number of brain scans together with
decisions previously taken on them (cancer found / cancer not found) -- the
\emph{training data}.
The learning algorithm then undergoes a \emph{training phase}. It attempts to
find an internal model that allows it to map the decisions to the brain scans,
given the training data. The model obtained from the training step is useful if
it performs well (generalizes) when applied to new data from outside the training
set -- in our example, when it can accurately diagnose brain cancer for new
brain scans. Such generalization is based on the premise that inputs that are
``closer'', in terms of the given \emph{features}, should lead to ``closer''
outputs.

The formal notion of ``closeness'' is a characteristic of the learner
algorithm being employed and determines how the algorithm generalizes the computation from
the given examples. Achieving good generalizations is the cornerstone of
machine learning and \emph{overfitting} (performing very well on training
inputs but very poorly on new inputs) is one of its major challenges.\levi{cut
these last couple of sentences}

More formally, in textbooks, courses and articles, Machine Learning is often
defined following the definition of Tom Mitchell~\cite{mitchell1997}:
\begin{quote}
	A computer program is said to learn from experience E with respect to some class of tasks T and performance measure P if its performance at tasks in T, as measured by P, improves with experience E.
\end{quote}
Therefore, it is said that to classify some patients into classes (e.g healthy and unhealthy), the task T, one have to define an algorithm that provides a model, such as an artificial neural network. The quality of this model is quantified by a measure P, for instance its accuracy while predicting the classes. This measure is then sent back to the algorithm, a new experience E, in order to choose or improve the model.
A machine learning tasks can be discussed and subdivided based on the elements
of the following equation:
\begin{equation*}
	f(\mathsf{\bf X}) = \mathsf{\bf y} + \xi
\end{equation*}
where $\mathsf{\bf X}$ is the $n \times d$ input matrix, containing $n$ samples
characterized by $d$ features, $\mathsf{\bf y}$ is the $n \times 1$ target
vector containing the classes of the $n$ samples and $\xi$ is a $n \times 1$
vector representing the noise. The goal is to approximate $f$ in order to
provide the best mapping between $\mathsf{\bf X}$ and $\mathsf{\bf y}$, given
some noise $\xi$. Indeed, the approximation has to map a $\mathsf{\bf X}$
containing noise, to a $\mathsf{\bf y}$ which may contain noise too. For
instance, uncontrolled conditions such as the room temperature and the exposure
time to this temperature can induce variations in the information contained in 
collected blood samples. Moreover, the $\xi$ term also contains the
approximation error when, for example, one tries to approximate a non-linear
function with a linear function.

The problem presented by the above equation is called \emph{supervised learning}, and can be roughly subdivided
in two popular problems: \emph{classification} and \emph{regression}.
When the target vector ${\bf y}$ is composed of categorical values (i.e. classes), then we have a
classification problem. The goal is to learn how to link instances or samples in
${\bf X}$ to a certain class (e.g. healthy patient or unhealthy patient).
However, if the target vector contains continuous values, we face a regression
problem (e.g. predict the body temperature of a patient given some clinical features of the
patient).

In some cases, ${\bf y}$ is not given and we have to find patterns in ${\bf X}$
``blindly.'' This is called an \emph{unsupervised} learning problem. Finding
clusters in ${\bf X}$, i.e. finding a ${\bf y}$ that has never been given, is such a
problem. For instance, one may want to group patients based on symptoms they have.

\emph{Reinforcement Learning} can be seen as
an intermediate problem where ${\bf y}$ is not given but the procedure is guided nevertheless. In RL, an agent has
to find a sequence of actions leading to a success. The fact that the sequence
leads to a success, i.e what would be in ${\bf y}$, is not known in advance, but
rewards are given to the agent in order for him to know if it follows a path to
success. In other words, the goal is, by providing rewards along the way, to
find the sequence of actions leading to the desired state. Typical examples can
be found in gaming, where an agent receives a reward when he wins the game. From
the different chains of actions that led him to a reward, the agent must generalize to find how to win that game.
\subsection{Requirements Engineering}

Software systems are developed over millions of lines of code, number of modules
and documents. The primary goal of the software system is to satisfy users by
developing the software that can meet their needs and expectations. This goal is
achievable by applying different methodologies and engineering techniques. One
of the key factor is to understand and identify the needs of users, also known
as, software requirements. Software requirement engineering is the process that
helps to determine the requirements in a systematic way to know what
functionalities the targeted system should have to fulfil user's needs. Formally
RE is defined as \cite{Zave:1997:CRE:267580.267581}:\\
\begin{quote}
``Requirements engineering is the branch of software engineering concerned with the real-world
goals for, functions of, and constraints on software systems. It is also concerned with the
relationship of these factors to precise specifications of software behavior, and to their evolution
over time and across software families."
\end{quote}
Software requirements play a key role in the success of a project. In the USA, a
survey was conducted over 8380 projects by 350 companies to know the project
failure rates. The report overall results showed only 16.2\% projects were
completed successfully and one-half (52.7\%) of projects met with challenges and
were completed with partial functionalities, time delays and over budget. Almost
31\% of the projects were never completed. The main cause told by the executive
managers was the poor requirement. The major problems were the lack of user
involvement (13\%), requirements incompleteness (12\%), changing requirements
(11\%), unrealistic expectations (6\%), and unclear objectives
(5\%).\cite{knuthwebsite} [6]\\
Software requirement engineering has mainly four phases; requirement
elicitation, requirement analysis, requirement documentation and requirement
verification \cite {Kotonya:1998:REP:552009}. Requirement elicitation
\cite{Coughlan:2002:ECR:2740304.2740444,Zowghi2005} helps to understand the
stakeholders needs, e.g. what features he wants in the software.
Requirement elicitation techniques are mostly derived by the social sciences,
organizational theory, knowledge engineering and practical experience. For
requirements elicitation, different techniques exist in the literature that
include interviews, questioners and ethnography etc.
Requirement analysis \cite{Nuseibeh:2000:RER:336512.336523} is the next step
after requirement elicitation. In this phase, software requirements are analyzed
to check conflicts and consistency of requirements. It is also makes sure that
the requirements are clear, complete and consistent. Furthermore, the agreed
requirements are documented. This documentation has a clear and precise
definition of the system functionalities. It also acts as an agreement between
stakeholders and developers. These functionalities and requirements are
documented usually as diagrams, mathematically formulae or natural languages.
These documents are used and iterated until the end of the projects.
System requirements are classified into businees requirments, user requirments,
functional requirements (FR) and non-functional requirements (NFR). Functional
requirements are the system requirements that include the main features and
characteristics of the desired system. Non-functional requirements are the
system properties and constraints \cite{Davis:1993:SRO:113586}. NFRs set the
criteria for judging the operation of the system e.g.
performance, availability and reliability etc.\levi{what about the business and
user requirements. Do we consider them?}
