\subsection{Requirements Management}

\subsubsection{Visualization}

Natural language requirement documents can be hard to comprehend and analyze. Similarly, stakeholders have to review and understand requirements for large and complex systems.  In these scenarios, basic information visualizations, like charts and graphs have been used in requirements engineering. These visualizations are usually applied to improve textual requirements with summarization that combined large amounts of information into a single representation for quick consumption by stakeholders\cite{Reddivari:2012}. Machine learning is of great value in discovering visualized groups of large numbers of requirements artifacts.\\

ReCVisu (Requirements Clustering Visualization) tool is presented in Reddivari \etal~\cite{Reddivari:2012} paper. ReCVisu, an exploration tool based on quantitative visualizations helps requirements engineers understand the nature of the requirements in a visual form. In ReCVisu, the dependence graph consists of requirements artifacts as nodes and the textual similarities as edges. The automatic grouping of requirements into clusters can help in areas such as uncovering the requirements structure, navigating around the requirements space, modularizing crosscutting concerns, and understanding requirements interactions and evolution.\newline

Pinqui \etal~\cite{Pinqui:2015} recognize the enormous volume of requirements as big data with which companies struggle to make strategic decisions early on. Therefore, they built a complete visual framework to filter requirements from stakeholders in a way that architects can make better insightful decisions. They suggest training a multi-class SVM model from domain-specific (mechanics, electronics, etc.) dictionaries and handbooks. Overall, the paper proposes a framework to go from management-oriented requirements to architecture-oriented requirements in which SVM is only applied in a small part of it. 

\subsubsection{Structuring Documents}

