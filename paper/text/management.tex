\subsection{Requirements Management}

\subsubsection{Visualization}

Natural language requirement documents can be hard to comprehend and analyze.
Similarly, stakeholders have to review and understand requirements for large and
complex systems.  In these scenarios, basic information visualizations, like
charts and graphs have been used in requirements engineering.
These visualizations are usually applied to improve textual requirements with
summarization that combined large amounts of information into a single
representation for quick consumption by stakeholders\cite{Reddivari:2012}.
Machine learning is of great value in discovering visualized groups of large
numbers of requirements artifacts\levi{I think you mean ``machine learning
is useful in grouping requirements for visualization purposes''}.\levi{which
kind of machine learning taks? Clustering?}\\

ReCVisu (Requirements Clustering Visualization) tool is presented in Reddivari
\etal~\cite{Reddivari:2012} paper. ReCVisu, an exploration tool based on
quantitative visualizations helps requirements engineers understand the nature
of the requirements in a visual form. In ReCVisu, the dependence graph consists
of requirements artifacts as nodes and the textual similarities as edges. The
automatic grouping of requirements into clusters can help in areas such as
uncovering the requirements structure, navigating around the requirements space,
modularizing crosscutting concerns, and understanding requirements interactions
and evolution.\newline

Pinqui \etal~\cite{Pinqui:2015} recognize the enormous volume of requirements as
big data with which companies struggle to make strategic decisions early on.
Therefore, they built a complete visual framework to filter requirements from
stakeholders in a way that architects can make better insightful decisions. They
suggest training a multi-class SVM model from domain-specific (mechanics,
electronics, etc.) dictionaries and handbooks. Overall, the paper proposes a
framework to go from management-oriented requirements to architecture-oriented
requirements in which SVM is only applied in a small part of it. \newline

Software requirements are mostly stated in natural text notations such as user
stories which is making it hard for people to develop an accurate mental image
of the most relevant entities and relationships. Lucassen
\etal~\cite{Lucassen:2016} introduced an automated method for visualizing
requirements at different levels of granularity. Their visualization method from
user stories consists of 1) the generation of an overview which provides a
general context for understanding the dataset:
\begin{itemize}
\item Extract a set of relevant concepts from the user stories and their relationships 
\item Calculate the semantic similarity by using skip-gram implementation word2vec
\item Utilize Ward’s clustering algorithm to group all the concepts according to their similarity 
\item Identify the concept which is most similar to the collection of concepts in a cluster
\item Generate inter-cluster relationships matrix
\item Visualization Drawing
\end{itemize}
2) zooming in and out mechanisms and 3) filtering techniques to reduce the
complexity of the data presentation. Possible anticipated applications of this
visualization are: discovering missing relationships between clusters that may
result in further user stories, teaching system functionality by exploring
simplified, manageable chunks, and analyzing expected system changes after
introducing new sets of user stories.

\subsubsection{Structuring Documents} 

Requirements of the system are usually presented in natural language documents. These documents require to be properly structured for a better overall understanding of the requirements. For this purpose, the document should be organized with independent sections which each one contains conceptually connected requirements\cite{Ferrari:2013}. Moreover, technical review is a usual way to guarantee the quality in natural language specifications. However, extensive and comprehensive specifications make it problematic for reviewers to find defects, especially consistency or completeness ones. Therefore, use of machine learning algorithms can support reviewers with their work by automatically classifying and clustering the information that is spread over many sections of many documents \cite{Ott:2013}.\\

Duan \etal~\cite{Duan:2007} used hierarchical automated clustering technique for detecting cross-cutting concerns as it is beneficial for the process of requirements analysis and architectural design. The reported experiments in this paper were supported by two tool sets, Poirot, a web-based tool designed to generate traces between various software engineering artifacts which was applied to compute similarity scores between requirements and a developed prototype tool, capable of reading structured requirements specification and generated similarity scores and then clustering requirements. \newline

Requirements engineering process results are usually documented in the natural language specifications. In most cases, these documents not only contain requirements but also some additional information such as explanations, summaries, and figures. As it is important to differentiate between relevant requirements and other auxiliary content it is often the case that requirements engineers manually label each element of the specification document. Winkler \etal~\cite{Winkler:2016} applied convolutional neural networks to automatically classify content elements of a natural language requirements specification as “requirement” or “information”. Their approach increases the quality of requirements specifications as it distinguishes important content for activities like test and etc. For converting natural language into a vector representation word2vec method is used. A set of 10000 content elements extracted from 89 requirements specifications of an industry partner used for training the network through the use of Tensorflow library using stochastic gradient descent. \newline





Ferrari \etal~\cite{Ferrari:2013}

Rauf \etal~\cite{Rauf:2011}

Ott \etal~\cite{Ott:2013}




