\subsection{Machine Learning}

Machine-Learning (\ML) \cite{mitchell:1997} is a range of algorithms to
approximate discover patterns in data. Historically, models and heuristics are
human-built exhaustive prescriptions of how a system should behave. \ML is grounded on different premises:
rather than relying on humans to input all the possible cases the system can
handle, the field attempts to extrapolate patterns from a representative
set of examples that illustrates expected behaviors. The way in which a
learning algorithm operates attempts to emulate the way in which humans learn:
from a set of examples, a general model for a behavior is induced.

Many learning algorithms exist, based on different visions of how learning
happens in practice~\cite{Domingos:2015}. All these algorithms have in common
the notion of \emph{features}. Features correspond to characteristics of what is
being learned and provide the grounds for the algorithm to abstract from the
complexities of the real world. Assume for example that an algorithm should
learn, based on a brain scan of a medical patient, to decide whether that
patient has brain cancer or not. A number of \emph{features} such as for example
the ``number of irregular objects in the scan'', the ``color of such objects'',
the ``disposition of such objects'' would be provided to the algorithm.
Additionally, the algorithm is provided with a number of brain scans together
with annotations that summarize decisions previously taken on them (in our
example cancer found / cancer not found). Such datasets are called the
\emph{training data}.
The learning algorithm then undergoes a \emph{training phase}. It attempts to
find an internal model that allows it to map the decisions to the brain scans,
given the training data. The model obtained from the training step is useful if
it performs well (generalizes) when applied to new data from outside the training
set -- in our example, when it can accurately diagnose brain cancer for new
brain scans. Such generalization is based on the premise that inputs that are
``closer'', in terms of the given \emph{features}, should lead to ``closer''
outputs.

The established literature in the domain (e.g.~\cite{mitchell:1997}) typically
considers three types of machine learning:

\subsubsection{\emph{Supervised learning: consists of learning a function using
training data including annotations of the outcome of the function to be learned
(e.g. patient John Doe with a certain number of physiological characteristics
was diagnosed with cancer). Supervised learning can be roughly subdivided in two
popular problems:
\emph{classification} and \emph{regression}. When the output of the function
being learned is composed of categorical values (i.e. classes), then we have a
classification problem. The goal is to learn how to link instances or samples
for a number of parameters to a certain class of values (e.g. healthy patient or
unhealthy patient). However, if the co-domain of the function being learned
contains continuous values, then we face a regression problem (e.g. predict the
body temperature of a patient given some clinical features of the patient).}}

\subsubsection{\emph{Unsupervised learning: In some cases, the output of the
function being learned is not given and we have to find patterns in the training data ``blindly.'' This is called an
\emph{unsupervised} learning problem. For instance, one may want to cluster
patients based on symptoms they have.}}

\subsubsection{\emph{Reinforcement Learning: can be seen as an intermediate
problem the co-domain of the function being learned is not given but the procedure is guided
nevertheless. In reinforcement learning, an agent has to find a sequence of
actions leading to a success. The fact that the sequence leads to a success, is
not known in advance, but rewards are given to the agent in order for it to know
if it follows a path to success. Typical examples can be found in gaming, where
an agent receives a reward when it wins the game. From the different chains of
actions that led him to a reward, the agent must generalize to find how to
improve at that same game.}}