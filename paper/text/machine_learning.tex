\subsection{Machine Learning}

\tahira{ annotations}


Machine-Learning (\ML) \cite{mitchell:1997} is a range of algorithm to
approximate functions and discover patterns in data.
Historically, models and heuristics are human-built exhaustive prescriptions of
how a system should behave. \ML is grounded on different premises:
rather than relying on humans to input all the possible cases the system can
handle, the field attempts to extrapolate patterns from a representative
set of examples that illustrate the expected behaviors. The way in which a
learning algorithm operates attempts to emulate the way in which humans learn:
from a set of examples, a general model for a behavior is induced.

Many learning algorithms exist, based on different visions of how learning
happens in practice~\cite{Domingos:2015}. All these algorithms have in common
the notion of \emph{features}. Features correspond to characteristics of what is
being learned and provide the grounds for the algorithm to abstract from the
complexities of the real world. Assume for example that an algorithm should
learn, based on a brain scan of a medical patient, to decide whether that
patient has brain cancer or not. A number of \emph{features} such as for example
the ``number of irregular objects in the scan'', the ``color of such objects'',
the ``disposition of such objects'' would be provided to the algorithm.
Additionally, the algorithm is fed a number of brain scans together with
decisions previously taken on them (cancer found / cancer not found) -- the
\emph{training data}.
The learning algorithm then undergoes a \emph{training phase}. It attempts to
find an internal model that allows it to map the decisions to the brain scans,
given the training data. The model obtained from the training step is useful if
it performs well (generalizes) when applied to new data from outside the training
set -- in our example, when it can accurately diagnose brain cancer for new
brain scans. Such generalization is based on the premise that inputs that are
``closer'', in terms of the given \emph{features}, should lead to ``closer''
outputs.

The formal notion of ``closeness'' is a characteristic of the learner
algorithm being employed and determines how the algorithm generalizes the computation from
the given examples. Achieving good generalizations is the cornerstone of
machine learning and \emph{overfitting} (performing very well on training
inputs but very poorly on new inputs) is one of its major challenges.\levi{cut
these last couple of sentences}

More formally, in textbooks, courses and articles, Machine Learning is often
defined following the definition of Tom Mitchell~\cite{mitchell1997}:
\begin{quote}
	A computer program is said to learn from experience E with respect to some class of tasks T and performance measure P if its performance at tasks in T, as measured by P, improves with experience E.
\end{quote}
Therefore, it is said that to classify some patients into classes (e.g healthy and unhealthy), the task T, one have to define an algorithm that provides a model, such as an artificial neural network. The quality of this model is quantified by a measure P, for instance its accuracy while predicting the classes. This measure is then sent back to the algorithm, a new experience E, in order to choose or improve the model.
A machine learning tasks can be discussed and subdivided based on the elements
of the following equation:
\begin{equation*}
	f(\mathsf{\bf X}) = \mathsf{\bf y} + \xi
\end{equation*}
where $\mathsf{\bf X}$ is the $n \times d$ input matrix, containing $n$ samples
characterized by $d$ features, $\mathsf{\bf y}$ is the $n \times 1$ target
vector containing the classes of the $n$ samples and $\xi$ is a $n \times 1$
vector representing the noise. The goal is to approximate $f$ in order to
provide the best mapping between $\mathsf{\bf X}$ and $\mathsf{\bf y}$, given
some noise $\xi$. Indeed, the approximation has to map a $\mathsf{\bf X}$
containing noise, to a $\mathsf{\bf y}$ which may contain noise too. For
instance, uncontrolled conditions such as the room temperature and the exposure
time to this temperature can induce variations in the information contained in 
collected blood samples. Moreover, the $\xi$ term also contains the
approximation error when, for example, one tries to approximate a non-linear
function with a linear function.

The problem presented by the above equation is called \emph{supervised learning}, and can be roughly subdivided
in two popular problems: \emph{classification} and \emph{regression}.
When the target vector ${\bf y}$ is composed of categorical values (i.e. classes), then we have a
classification problem. The goal is to learn how to link instances or samples in
${\bf X}$ to a certain class (e.g. healthy patient or unhealthy patient).
However, if the target vector contains continuous values, we face a regression
problem (e.g. predict the body temperature of a patient given some clinical features of the
patient).

In some cases, ${\bf y}$ is not given and we have to find patterns in ${\bf X}$
``blindly.'' This is called an \emph{unsupervised} learning problem. Finding
clusters in ${\bf X}$, i.e. finding a ${\bf y}$ that has never been given, is such a
problem. For instance, one may want to group patients based on symptoms they have.

\emph{Reinforcement Learning} can be seen as
an intermediate problem where ${\bf y}$ is not given but the procedure is guided nevertheless. In RL, an agent has
to find a sequence of actions leading to a success. The fact that the sequence
leads to a success, i.e what would be in ${\bf y}$, is not known in advance, but
rewards are given to the agent in order for him to know if it follows a path to
success. In other words, the goal is, by providing rewards along the way, to
find the sequence of actions leading to the desired state. Typical examples can
be found in gaming, where an agent receives a reward when he wins the game. From
the different chains of actions that led him to a reward, the agent must generalize to find how to win that game.