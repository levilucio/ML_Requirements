\subsection{Requirements Validation}

Validation is to meant to guarantee that requirements are reflecting the
stakeholders' needs, confirm the quality of the system, its consistency, and
traceability.

\subsubsection{Traceability}

In requirements traceability the emphasis is on the ability to track the
lifecycle of requirements and their links with other artifacts. The main barrier
to ensure traceability is the effort required for building and maintaining the
links between those artifacts. Researchers have thus tried to apply
machine learning and automated tools for facilitating the establishment of such
links\cite{Gervasi:2011}.
Traceability is tackled in the research mainly through the use of machine
learning classification as well as reinforcement learning methods.

Gervasi \etal have investigated in~\cite{Gervasi:2011} what can be learned from
links that are already established. They build classifiers as a mean to develop models
of tracing that can then be interpreted by humans in order to understand how
requirement tracing is done in practice. Their purpose is to revise existing models of
hard-coded traceability tools such as VSM. They used a publicly-available
dataset of requirements including traceability information, originally based on
the CM-1 project by the NASA Metrics Data Program. Their approach has the following
steps: apply preprocessing techniques and transfer requirements into a
vector of features, from which a set of classification cases is derived by
joining a high-level requirement and one low-level requirement. For every such
pair a \textit{link} or \textit{nolink} tag is added, based on whether that
particular pair was a true link in the original dataset or not. Finally, they
use use the dataset to train and test two different classifiers from the WEKA
collection: a Naive Bayesian classifier and the J48 decision-tree classifier.

Sultanov \etal~\cite{Sultanov:2013} find traceability candidates from
high-level to low-level requirements by the use of reinforcement
learning. They use textual high and low-level requirements documents as input and try to
find the candidate traces. Their technique demonstrated statistically
significantly better results than the Information Retrieval
technique\levi{what is this technique? Reference?}.





