\subsection{Requirements Validation}

\subsubsection{Traceability}

Validation is to guarantee that requirements are reflecting stakeholders' needs,
confirm the quality of the system, consistency, and traceability.
 One of the definitions for requirements traceability is given by
 \cite{Gotel:1994}:
\begin{displayquote} 
“Requirements traceability refers to the ability to describe and follow the life
of a requirement, in both a forward and backward direction (i.e., from its
origins, through its development and specification, to its subsequent deployment
and use, and through periods of on-going refinement and iteration in any of
these phases).”
\end{displayquote}
Based on this definition the emphasis is on the ability to track the life of
requirements and their established links within other artifacts. However, the
main barrier assures traceability is the needed effort for building and
maintaining the links between those artifacts. That is why many research has
tried to apply machine learning and automated tools for facilitating the
establishment of links\cite{Gervasi:2011}.
Traceability tackled in the research mainly by the use of machine learning
classification and reinforcement learning methods.\\

Gervasi \etal~\cite{Gervasi:2011} investigate what can be learned from links that are already established. They build classifiers as a mean to develop models of tracing that can then be interpreted by humans to understand how requirement tracing is done in practice. Their purpose is to revise the existing models of hard-coded traceability tools such as VSM. They used a publicly-available dataset of requirements with traceability information, originally based on the CM-1 project by the NASA Metrics Data Program. Their approach has 5 steps: 1) tokenize and stem requirements and removing stopwords 2) derive two features from each term t in the vocabulary, one for the occurrence of t in a high-level requirement and one for the similar occurrence in a low-level requirement 3) transfer requirements into a vector of features 4) From these vectors derive set of classification cases by joining one high-level requirement and one low-level requirement and adding a classification of \textit{link} or \textit{nolink} based on whether that particular pair was a true link in the original dataset, or not 5) finally, use the dataset to train and test two different classifiers from the WEKA collection, a Naive Bayesian classifier, and the J48 decision-tree classifier.\newline

Sultanov \etal~\cite{Sultanov:2013} finds traceability candidates from
high-level to low-level requirements by the use of reinforcement
learning. They used textual high and low-level requirements documents as an input and try to
find the candidate traces. Their technique demonstrated statistically
significantly better results than the Information Retrieval technique.





