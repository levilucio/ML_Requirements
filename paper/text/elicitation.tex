\subsection{Requirements Elicitation and Discovery}

The manual process of requirement elicitation is expensive in terms of effort
and resources. A project's success majorly depends on the precise identification
of stakeholder's expectations and requirements for their desired system. A
possibility to do requirements elicitation is to mine available datasets, e.g.
social media, requirement documents or  app store reviews. This mining process
is performed with help of different techniques such as \NLP and text mining
\cite{Hollis2017}\cite{dong2010}. The latest trend for identifying
user requirements is to mine data obtained from platforms such as Twitter,
Google Play Store or the Apple Store, by applying \ML techniques. While user
reviews are not structured requirements, they contain useful information
coupled with extra information and noise, which makes manual requirement
elicitation a challenging task. Automated requirement elicitation is desirable
in these cases to significantly reduce time, effort, and cost. Elicitation from
external platforms is mainly a ML \emph{classification} task: given a set of
information we wish to identify which parts are it constitute requirements. In
the literature we surveyed we notice that also \emph{clustering} is used in
auxiliary tasks.

Guzman \etal \cite{Guzman:2017} proposed the ALERTme approach for classifying,
grouping and ranking tweets during software evolution. Many users shared their
opinions about various software on Twitter. The large amount of datasets made it
hard to manually identify tweets that contained user requirements. The proposed
methodology classified tweets as improvement requests or not, using the
\emph{Naive Bayes} algorithm. This was the first study of its kind.
The classifier was trained according to the following steps: 1) conversion of
the pre-processed tweets into a \VSM model, 2) training of a classifier on a set
of manually annotated tweets, 3) classifying tweets in categories by using the
trained classifier. Furthermore, improvement requests were considered during
tweet grouping, which helped in sort and summarizing the requests. The
results summarization process contained highly ranked tweets based on
parameters such as the number of ``likes'', sentiment, number of shares among
others.

Williams \etal \cite{Williams:2017} operated a similar study on tweets in order
to classify them as user requirements. The authors used basic pre-processing
techniques and applied the \VSM on the data. For the learning process, manually
annotated (labelled) tweets and the \emph{Naive Bayes} algorithm was used for
performing the classification. Based on the results, the authors claime in their
work that software tweets are neutral in nature -- meaning sentiment analysis
did not influence the outcome of the ML algorithm. Also, the work showed
improved results when compared comparison to \cite{Guzman:2017}.
The study used 4000 randomly selected tweets from ten different software
including Microsoft Visual Studio, Google Chrome and Instagram.

Jiang \etal \cite{Jiang:2014} mined user reviews from app stores for discovering
evolutionary requirements. The authors first extracted opinions about software
features from reviews. For automated opinion identification, a syntactic
relation-based propagation approach was used for extracting targets and
sentiment words iteratively, using known and extracted words. Afterwards, the
authors applied k-mean clustering for opinion categorization. The proposed
system also helped developers deciding on requirements related to software
revenue, by considering economic factors. The work used two datasets of online
reviews: one from the Karplersky internet security 2011 software package (from Amazon) with 380
reviews; the other one comprising 461 reviews for the TuneIn Radio Pro V3.6
mobile app (from the app store\levi{Apple's?}).

Lange \etal\cite{Douglas:S2008} mapped the software requirement elicitation
process onto an existing military tool \emph{skiweb}. Skiweb is used to make
decisions about what actions need to be taken on a military
command\levi{during action?}.
Different users posted and updated events and information using this tool. The goal of
adding learning capabilities was to find additional information relevant to user
posts. The proposed recommender system used supervised \emph{Naive Bayes}
algorithm to classify text documents in order to find related requirements to
the post. Furthermore, it used topic modeling to identify the key stakeholders
and suggested them requirements for further analysis according to their
interest\levi{which analysis?}. This study used an internal organizational
dataset Skiweb Data such as wiki and blogs etc.\levi{this paragraph is unclear,
see with Tahira}

Jha \etal \cite{Jha:2017} discovered user requirements by mining app store
reviews. The requests were classified into three categories; \emph{bugs},
\emph{features} and \emph{junk}. The methodology proposed by the authors used
Naive Bayes and \SVM algorithms. The distinction between types of sentences was
identified by frame semantics \emph{explain frame semantics}, instead of text
classification methods. For each review frames were generated, rather than for
words. Due to small number of features, a slower\levi{slower than what?}
dimensional model was produced with enhanced prediction capabilities. The
study combined existing datasets from past studies and reviews for iOS apps
including CreditKarma, Fitbit, and Gmail.

Maalej \etal presents in \cite{Maalej} a study on how to classify app reviews as
bug reports, feature requests, user experiences or ratings. The authors used a
\emph{Naive Bayes} algorithm after comparing several algorithms for
classification. They also highlight that binary classifiers performed better
than multi classifiers\levi{what is the difference betwee binary and multi}. It
used meta model to enhance the classification performance e.g.
ratings, tense, and sentiment scores etc. A dataset of 4400 manually annotated
reviews from Google Play Store and the Apple App Store were used for the study.

Herrera \etal \cite{Castro-Herrera:2009} built a recommender system to manage a
the participation of a number of stakeholders in the requirements
elicitation and prioritization process. In this system, stakeholders could work collaboratively
to transform their needs into sets of articulated and prioritized requirements.
The system automatically generates specialized topics for building forums for
stakeholder collaboration and discussion. The stakeholders' interests are
extracted from their user profiles, which also helped in creating
recommendations according to the interest of a community of similar stakeholders.\levi{is this the
recommender part?} In order to identify topics, an unsupervised agglomerative
clustering algorithm was applied to unstructured data\levi{what's the
difference?}. The proposed system analyzed online datasets\levi{which datasets?}
that were gathered from stakeholders in natural language.
The evaluation dataset was a collection of 36 feature requests created by
graduate-level students for an Amazon-like student web-portal system.






