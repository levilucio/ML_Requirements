
\subsection{Requirements Elicitation and Discovery} The manual process of requirement elicitation is time-consuming and required efforts, and sources. The project
success majorly depends upon the precise identification of stakeholder expectations and requirements for their desired system. Poor elicitation can lead to over budget and time issues. Requirements elicitation can be done by mining the available dataset e.g., social media, documents, and gathering and analyzing information available on the internet, publically or privately available data. In short, gathering data from different means and analyze them for requirements elicitation.  The latest trend for identification of user requirement is mining social media like twitter, google store, and app store data. These user reviews are not structured requirement and contained other information too such as praise, dislikes, bug report etc.  The unstructured, extra information and noise for elicitation from the above-described sources makes the manual process difficult and challenging. The automated analysis of requirement elicitation is helpful and can significantly reduce time, effort, and cost. This is ML classification task: give the set of information and
identifying it as a requirement or not.
\\

\subsubsection{Internal vs External\levi{internal vs external is not explained}}
Guzman ~etal \cite{Guzman:2017} proposed ALERTme approach for classifying,
grouping and ranking tweets in software evolution process. For identification of
evolutionary requirement supervised classification performed on tweets as
improvement request or not using Naïve Bayes algorithm. It was the first kind of
this study that finds user requirements from tweets for software. Williams \etal
\levi{use the \etal command} \cite{Williams:2017} performed a similar study on
tweets for classifying as user requirement or not. The study gave state of the art results using Naïve Bayes
and VSM algorithm. Different techniques were implemented for feature extraction
that includes stopword removing, sentimental analysis, stemming\levi{bring to
machine learning section}, and Bag of words. For learning process maually annotated tweets were used. However, results showed that
software tweets are neutral in nature and sentiment analysis did not influence ML algorithm.\levi{instances are manually
classified tweets}\\

Research on mining user reviews in mobile application (app) stores has
remarkably advanced in the past few years. Jiang \etal \cite{Jiang:2014}
presented an optimized method for discovering the evolutionary requirements for
developers. This method clustered opinion expressions in form of a macro network
topology, and combine polarized sentimental analysis with the economic factors
to decide evolutionary requirements. The dataset used POS tagger and parser with
additional defined rules for feature extraction. Douglas S. ~etal
\cite{Douglas:S2008} mapped software requirement elicitation process to an
existing military tool skiweb that used for making the decision. The proposed
methodology used supervised Naive Bayes to classify text document to find
related requirements. Furthermore, recommendation system used topic modeling to
identify the key stakeholder for which that requirement is important and allow
for the analysis. Jha ~etal \cite{Jha:2017} Classified application store reviews
and discover user requirement as feature request. The data was classified into
three categories bugs, features, and junk. For classification, SVM and Naïve
Bayes algorithm was used. The study used frame semantic analysis that genralize more abstract contexts. This technique produced lower \levi{slower?} dimensional model
with smaller number of features, that help to enhance the prediction capabilities.Herrera\etal\cite{Castro-Herrera:2009} proposed a tool to scale up the support 
for large-scale systems requirements processes in comparison to traditional requirements elicitation process. The large oftware systems are complex due to this 
it is hard to manage stakeholders and identify the relevant stakeholders for further discussions.
The proposed recommender framework used feature requests of different projects in form of natural language statements. 
An unsupervised clustering techniques used to identify cohesive themes out of these features. Then tool constructed user
profiles according to the interests of the stakeholders in each of these themes for recommender system. Stakeholders needs (features) are initially preprocessed by
removing common (stop) terms, and computation of tf-id for each term. 

\subsubsection{Prioritization of Requirements} 

\subsubsection{Text mining} 
\levi{don't understand this sentence} Kaiya ~etal \cite {Kaiya:2010} proposed a
tool to improve the domain knowledge ontology for requirement elicitation by using web mining and NLP
technique. It helped to mine the general
concepts to ontology for elicited requirements.
Hollis \etal\cite{Hollis2017} proposed an initial study to
automate requirement elicitation in Agile environment. By providing the list of
words and loosely formatted list of requirements. The proposed methodology
applied text mining technique on recorded conversation of the stakeholder and
developer conversation. Dong ~etal \cite{dong2010} also applied text mining on
different form of document and resources from internet for gathering
requirement. The system applied data preprocessing as word segmentation and stop
words removal and build up the VSM model. \levi{not clear what the \ML
contribution is here. Model, features and instances are missing.} {Clealand Huang:2008} provide semi-automatically manage broad stakeholder participation in the requirements elicitation and prioritization process. 
The proposed system analyzed datset i.e. gathered from stakeholders and automatically generated highly specialized topical forums i.e. themes identification, and assign it to stakeholders accordingly.
On these forums stakeholder can work collaboratively to transform statements of need into sets ofarticulated and prioritized requirements. For themes identification unsupervised clustering
techniques applied to massive amounts of unstructured or semi-structured data. 


