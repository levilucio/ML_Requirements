
\subsection{Requirements Elicitation and Discovery}

The manual process of requirement elicitation is time-consuming and requires
efforts, and sources\levi{which sources?}. A project's success majorly depends
upon the precise identification of stakeholder expectations and requirements for their desired
system. Poor elicitation can lead to over budget and time issues.
\levi{A possibibility to do?} Requirements elicitation can be done by mining
available datasets e.g., social media, documents, and gathering and analyzing information available on the internet,
publically or privately available data. In short, gathering data from different
means and analyzing it for requirements elicitation.  The latest trend for
identifiying of user requirement is to mine social media coming from the likes
if twitter, google store, and app stores. These user reviews are not
structured requirements and contain other information too such as praise,
dislikes or bug reports. The unstructured, extra information and noise
presence in the above-described sources makes manual requirements
elicitation difficult and challenging. Automated requirement elicitation
is helpful in these cases and can significantly reduce time, effort, and cost.
This is ML classification task \emph{what about clustering?}: give the set of
information and identifying it as a requirement or not.
\\

Guzman \etal \cite{Guzman:2017} have proposed the ALERTme approach for
classifying, grouping and ranking tweets in\levi{during the?} software evolution
process. For identifying evolutionary requirement supervised classification
has been performed on tweets to classify them as improvement requests or not,
using the Naïve Bayes algorithm. This was the first this study of its kind
performed on software tweets. For tweets preprocessing tokenization, conversion of all the texts into lower case, extracting n-gram with a one to three words length, removing stopwords and stemming were applied. 
After this classifier was trained with following steps: 1) conversion of preprocessed tweets into a VSM model, 2) train a classifier on a set of manually annotated tweets, 3) predict the tweets categories using a trained classifier.\levi{features, datasets?}\\

Williams \etal \cite{Williams:2017} performed a similar study on tweets for
classifying them as user requirements or not. The study gave state of the art
results\levi{what does this mean?} using Naïve Bayes and VSM
algorithm\levi{explain what VSM is}.
Different techniques were implemented for feature extraction that includes stopword removing, sentimental analysis,
stemming\levi{bring to machine learning section}, and Bag of words.
For learning process maually annotated tweets were used. However, results showed
that software tweets are neutral in nature and sentiment analysis did not
influence ML algorithm.\\

Research on mining user reviews in mobile application (app) stores has
remarkably advanced in the past few years.\levi{remove} Jiang \etal
\cite{Jiang:2014} presented an optimized method\emph{what is optimized?} for
discovering the evolutionary requirements for developers. This method clustered opinion expressions in form of a macro network
topology, and combine polarized sentimental analysis with the economic factors
to decide evolutionary requirements. The dataset used POS tagger and parser with
additional defined rules for feature extraction.\levi{don't understand this
sentence. What dataset was used?}\\

Douglas.S \etal\cite{Douglas:S2008} mapped software requirement elicitation process to an
existing military tool skiweb that used for making the decision. The proposed
methodology used the supervised Naive Bayes algorithm to classify text document
to find related requirements. Furthermore, recommendation system used topic modeling to
identify the key stakeholder for which that requirement is important and allow
for the analysis.\levi{features, datasets?}\\

Jha \etal \cite{Jha:2017} classified application store reviews and discover user
requirement as feature request\levi{discovered user requirements by mining
feature requests?}. The data was classified into three categories bugs,
features, and junk. For classification, SVM and Naïve Bayes algorithm was used.
The study used frame semantic analysis that genralize more abstract contexts.
This technique produced slower dimensional model with smaller number of
features, that help to enhance the prediction capabilities.\levi{features,
datasets?}\\


\subsubsection{Text mining}

 
 	Herrera \etal \cite{Castro-Herrera:2009} provide semi-automatically manage broad stakeholder participation in the requirements elicitation and prioritization process. 
The proposed system analyzed datset i.e. gathered from stakeholders and automatically generated highly specialized topical forums i.e. themes identification, and assign it to stakeholders accordingly.
On these forums stakeholder can work collaboratively to transform statements of need into sets of articulated and prioritized requirements. For themes identification unsupervised clustering
techniques applied to massive amounts of unstructured or semi-structured data. Then tool constructed user profiles according to the interests
of the stakeholders in each of these themes for recommender system. Stakeholders
needs (features) are initially preprocessed by removing common (stop) terms, and
computation of tf-id for each term. The dataset was  small collection of featur requests created by 36 graduate level students for an Amazon-like student web-portal system.\\

	Hollis \etal \cite{Hollis2017} proposed an initial study to automate requirement elicitation in Agile environment by providing the list of
words and loosely formatted list of requirements. The proposed methodology
applied text mining technique on recorded conversation of the stakeholder and
developer conversation. It was a short paper and did not provide details.\\

	Dong ~etal \cite{dong2010} also applied text mining on
different form of document and resources from internet for gathering
requirement. The system applied data preprocessing as word segmentation and stop
words removal and build up the VSM model. Kaiya \etal \cite {Kaiya:2010} proposed a
tool to improve the domain knowledge ontology for requirement elicitation by using web mining and NLP
technique.It helped to mine the general
concepts to ontology for requirements elicitation.\\
  \levi{not clear what the \ML
contribution is here. Model, features and instances are missing.} 


