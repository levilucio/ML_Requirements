\subsection{Requirements Elicitation and Discovery} The manual process of
requirement elicitation is time consuming and required efforts. The project
success majorly depends upon right identification of stakeholder expectations
and requirements. The latest trend for identification of user requirement is
mining social media like twitter, google store, and app store data.\levi{add
citation} These user reviews are not structured requirement and contained other
information too such as praise, dislikes, bug report etc. The manual analysis of these large user
reviews dataset and identify the user requirements hard. The automated analysis
of requirement elicitation is helpful and can significantly reduce time and
cost. This is ML classification task: give the set of information and
identifying it as requirement or not.\levi{we need to speak about other sources
of requirements other than social media}\\

\cite{Cleland-Huang:2008}, \cite{Kaiya:2010}, \cite{Castro-Herrera:2009}
% <<<<<<< HEAD
% \subsubsection{Internal vs External}
% Guzman ~etal \cite{Guzman:2017} proposed ALERTme approach for classifying,
% grouping and ranking tweets in software evolution process. For identification of
% evolutionary requirement supervised classification performed on tweets as
% improvement request or not using Naïve Bayes algorithm. It was the first kind of
% this study that finds user requirements from tweets for software. Williams ~etal
% \cite{Williams:2017} performed a similar study on tweets for classifying as user
% requirement or not. The study gave state of the art results using Naïve Bayes
% and VSM algorithm. Different techniques were implemented for feature extraction
% that includes stopword removing, sentimental, stemming, and Bag of words.
% However, results showed that software tweets are natural and sentiment analysis
% did not influence ML algorithm.\\
% 
% Research on mining user reviews in mobile application (app) stores has
% remarkably advanced in the past few years. Jiang ~etal \cite{Jiang:2014}
% presented an optimized method for discovering the evolutionary requirements for
% developers. This method clustered opinion expressions in form of a macro network
% topology, and combine polarized sentimental analysis with the economic factors
% to decide evolutionary requirements. The dataset used POS tagger and parser with
% additional defined rules for feature extraction.                   Jha~etal
% \cite{Jha:2017} Classified application store reviews and discover user
% requirement as feature request. The data was classified into three categories
% bugs, features, and junk. For classification, SVM and Naïve Bayes algorithm was
% used. The study used frame semantic analysis for feature identification, this
% technique produced slower dimensional model with smaller number of features.
% Douglas S.~etal \cite{Douglas:S2008} mapped software requirement elicitation
% process to an existing military tool skiweb that used for making the decision.
% The proposed methodology used supervised Naive Bayes to classify text document
% to find related requirements. Furthermore, recommendation system used topic
% modeling to identify the key stakeholder for which that requirement is important
% and allow it for analysis.  [47]. \\
% 
% \subsubsection{Prioritization of Requirements} 
% 
% \subsubsection{Text mining}
% Kaiya ~etal \cite {Kaiya:2010} proposed a tool to improve the ontology of domain
% knowledge for requirement elicitation by using web mining and NLP technique. It
% helped to mine the general concepts to ontology for elicitating requirements.
% Hollis ~etal   \cite{Hollis2017} proposed an initial study to automate
% requirement elicitation in Agile environment. By providing the list of words and
% loosely formatted list of requirements. The proposed methodology applied text
% mining technique on recorded conversation of the stakeholder and developer
% conversation. Dong ~etal \cite{dong2010}  also applied text mining on different
% form of  document and resources from internet for  gathering requirement. The
% system applied data preprocessing as word segmentation and stop words removal
% before building up the VSM model.
% =======

Guzman ~etal \cite{Guzman:2017} proposed ALERTme approach for classifying,
grouping and ranking tweets in software evolution process. For identification of
evolutionary requirement supervised classification performed on tweets as
improvement request or not using Naïve Bayes algorithm. It was the first kind of
this study that finds user requirements from tweets for software. Williams ~etal
\levi{use the \etal command} \cite{Williams:2017} performed a similar study on
tweets for classifying as user requirement or not. The study gave state of the art results using Naïve Bayes
and VSM algorithm. Different techniques were implemented for feature extraction
that includes stopword removing, sentimental analysis, stemming\levi{bring to
machine learning section}, and Bag of words. However, results showed that
software tweets are neutral in nature and sentiment analysis did not influence ML algorithm.\levi{instances are manually
classified tweets}\\

Research on mining user reviews in mobile application (app) stores has
remarkably advanced in the past few years. Jiang ~etal \cite{Jiang:2014}
presented an optimized method for discovering the evolutionary requirements for
developers. This method clustered opinion expressions in form of a macro network
topology, and combine polarized sentimental analysis with the economic factors
to decide evolutionary requirements. The dataset used POS tagger and parser with
additional defined rules for feature extraction. Douglas S. ~etal
\cite{Douglas:S2008} mapped software requirement elicitation process to an
existing military tool skiweb that used for making the decision. The proposed
methodology used supervised Naive Bayes to classify text document to find
related requirements. Furthermore, recommendation system used topic modeling to
identify the key stakeholder for which that requirement is important and allow
for the analysis. Jha ~etal \cite{Jha:2017} Classified application store reviews
and discover user requirement as feature request. The data was classified into
three categories bugs, features, and junk. For classification, SVM and Naïve
Bayes algorithm was used. The study used frame semantic analysis for feature
identification, this technique produced slower \levi{slower?} dimensional model
with smaller number of features. [47]. \\

\subsubsection{Prioritization of Requirements} 

\subsubsection{Text mining} 
\levi{don't understand this sentence} Kaiya ~etal \cite {Kaiya:2010} proposed a
tool to improve the ontology of domain knowledge for requirement elicitation by using web mining and NLP
technique. It helped to mine the general
concepts to ontology for elicited requirements.
Hollis ~etal               \cite{Hollis2017} proposed an initial study to
automate requirement elicitation in Agile environment. By providing the list of
words and loosely formatted list of requirements. The proposed methodology
applied text mining technique on recorded conversation of the stakeholder and
developer conversation. Dong ~etal \cite{dong2010}  also applied text mining on
different form of  document and resources from internet for  gathering
requirement. The system applied data preprocessing as word segmentation and stop
words removal before building up the VSM model. \levi{not clear what the \ML
contribution is here. Model, features and instances are missing.}
