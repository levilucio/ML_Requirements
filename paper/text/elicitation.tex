
\subsection{Requirements Elicitation and Discovery}

The manual process of requirement elicitation is time-consuming and requires
efforts and resources. A project's success majorly depends
upon the precise identification of stakeholder expectations and requirements for their desired
system. Poor elicitation can lead to over budget and time issues.
A possibility to do requirements elicitation can be mining
available datasets e.g. social media, documents. Also, gathering and analyzing information available on the internet,
publically or privately available dataset. In short, gathering data from different
sources and analyzing it for requirements elicitation.  The latest trend for
identifying user requirement is to mine social media coming from the twitter, google store, and app stores. These user reviews are not
structured requirements and contain other information too such as praise,
dislikes or bug reports. The unstructured, extra information and noise
presence in the above-described sources makes manual requirements
elicitation difficult and challenging. Automated requirement elicitation
is helpful in these cases and can significantly reduce time, effort, and cost.
This is ML classification and clustering task give the set of
information and identifying it as a requirement or not.\\

Guzman \etal \cite{Guzman:2017} have proposed the ALERTme approach for
classifying, grouping and ranking tweets during the software evolution
process. For identifying evolutionary requirement supervised classification
has been performed on tweets to classify them as improvement requests or not,
using the Naïve Bayes algorithm. It was the first this study of its kind
performed on software tweets. For tweets preprocessing tokenization, conversion of all the texts into lower case, extracting n-gram with a one to three words length, removing stopwords and stemming were applied. 
After this classifier was trained with following steps:  1) conversion of preprocessed tweets into a VSM model, 2) train a classifier on a set of manually annotated tweets, 3) predict the tweets categories using a trained classifier. A data set of 68108 tweets i.e. collection of two-month software tweets of Spotify, Dropbox, and Slack was used.

Williams \etal \cite{Williams:2017} performed a similar study on tweets for
classifying them as user requirements or not. The study improved
results in comparison to the existing studies using Naïve Bayes and VSM
algorithm. Different techniques were implemented for feature extraction that includes stopword removing, sentimental analysis,
stemming\levi{bring to machine learning section}, and Bag of words.
For the learning process, manually annotated tweets were used. However, results showed
that software tweets are neutral in nature and sentiment analysis did not
influence the ML algorithm. It used 4000 randomly selected tweets from ten different software.\\

 Jiang \etal \cite{Jiang:2014} presented a study on mining user reviews in mobile application (app) stores for
discovering the evolutionary requirements for developers. This method clustered opinion expressions in the form of a macro network
topology, and combine polarized sentimental analysis with the economic factors
to decide evolutionary requirements. It used two datasets of online reviews one was from software package Karplersky internet security 2011 from Amazon with 380 reviews, and mobile app of TuneIn Radio Pro V3.6 from the app store with 461 reviews.\\

Lange \etal\cite{Douglas:S2008} mapped software requirement elicitation process to an
existing military tool skiweb that used for making the decision. The proposed
methodology used the supervised Naive Bayes algorithm to classify the text document
to find related requirements. Furthermore, the recommender system used topic modeling to
identify the key stakeholder for which that requirement is important and allow
for the analysis. This study used internal organizational dataset Skiweb Data i.e. wiki, blogs etc.\\

Jha \etal \cite{Jha:2017} discovered user requirements by mining
feature requests from the app store. The data were classified into three categories bugs,
features, and junk. For classification, SVM and Naïve Bayes algorithm were used.
The study used frame semantic analysis that generalize more general contexts.
This technique produced a slower dimensional model with a smaller number of
features which help to enhance the prediction capabilities. It combined the existing dataset from the past studies and iOS apps reviews including CreditKarma, FitBit, and Gmail.

Maalej \cite{Maalej} presented a classification study of app reviews into bug reports, feature requests, user experiences, and
ratings. The binary classifier Naive Bayes performed best than other algorithms also it showed binary classifier are suitable for predicting the than multi-class classifiers. Different review feature such as ratings, tense, sentiment scores, and length enhanced the classification performance. A dataset of 4400 manually annotated reviews from Google Play and the Apple app store was used. \\

Herrera \etal \cite{Castro-Herrera:2009} builded a semi-automated recommender system
to manage broad stakeholder participation in the requirements elicitation
and prioritization process. On these forums, stakeholders can work collaboratively to transform statements of
need into sets of articulated and prioritized requirements. The proposed system analyzed online dataset i.e. gathered
from stakeholders and automatically generated highly specialized topics for generating different forums. User profiles are created as a next step that depicts stakeholders interest. These profiles also help to create recommendations according to the interests of a community of similar stakeholders. For topics
identification, an unsupervised clustering technique is applied to unstructured or semi-structured data. 
For the evaluation dataset was a collection of feature requests. 36 graduate-level students created these requests for an Amazon-like student web-portal system.\\


\subsubsection{Text mining}\levi{add to preprocessing}


Hollis \etal \cite{Hollis2017} proposed an initial study to automate requirement
elicitation in Agile environment by providing the list of words and loosely
formatted list of requirements. The proposed methodology applied text mining
technique on recorded conversation of the stakeholder and developer
conversation. It was a short paper and did not provide details.\\

Dong \etal \cite{dong2010} also applied text mining on different form of
document and resources from internet for gathering requirement. The system
applied data preprocessing as word segmentation and stop words removal and build
up the VSM model. Kaiya \etal \cite{Kaiya:2010} proposed a tool to improve the
domain knowledge ontology for requirement elicitation by using web mining and
NLP technique. It helped to mine the general concepts to ontology for
requirements elicitation.


