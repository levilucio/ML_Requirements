\section{Contributions}

\levi{summarize preprocessing steps and feature calculation}
\levi{disclaimer: sometimes features are presented sometimes not}
\levi{?} Stakeholders needs
(features)\levi{features in this case are just words} are initially preprocessed
by removing common (stop) terms, and computation of tf-id \levi{explain tf–idf,
maybe extract it to a common section, in general explain that features are
words and the instances are tweets/reviews/\ldots} for each term.

Requirements are written in the natural language such as documents or reviewing some application. These can be in a variety of forms from a list of individual words, sentences, multiple paragraphs, short texts with special characters etc. Before applying a machine learning algorithm on them different steps employed to transform words into features a.k.a preprocessing. Text preprocessing relies majorly on pre-built dictionaries, databases, and rules. The common preprocessing steps in our literature include capitalization, tokenization, lemmatization, stop words removal, stemming, part of speech (POS). Tokenization is the process of splitting paragraphs into sentences, or sentences into words. Capitalization brings everything to lower case for simplicity. Stop words removal removes all connecting words such as and, the etc by comparing the text to a list of stopwords. POS takes text and assigns each part of speech to each word that helps to build more understanding of a text. Stemming is a process where words are reduced to a root by removing the unnecessary suffix e.g. eating after stemming is eat. Lemmatization is an alternative approach of stemming is able to capture canonical forms based on a word's lemma. It uses part of speech and  WordNet’s lexical database of English for removing inflection. The word better, stemming will fail to provide any lemma but lemmatization would result better -> good. 

 Another way to extract features from the text is Bag of Words (BoW). It is a model used in natural language processing and categorizes documents based on a vocabulary of words and occurrence of words. The common used NLP method is vector space modeling (VSM). It is a way to represent documents into multidimensional space for information retrieval and documents classification and clustering. For example, if you give a query VSM will find relevant items or term from a corpus.

Hollis \etal \cite{Hollis2017} proposed an initial study to automate requirement
elicitation in Agile environment by providing the list of words and loosely
formatted list of requirements. The proposed methodology applied text mining
technique on recorded conversation of the stakeholder and developer
conversation. It was a short paper and did not provide details.\\

Dong \etal \cite{dong2010} also applied text mining on different form of
document and resources from internet for gathering requirement. The system
applied data preprocessing as word segmentation and stop words removal and build
up the VSM model. Kaiya \etal \cite{Kaiya:2010} proposed a tool to improve the
domain knowledge ontology for requirement elicitation by using web mining and
NLP technique. It helped to mine the general concepts to ontology for
requirements elicitation.

\subsection{Requirements Elicitation and Discovery}

The manual process of requirement elicitation is expensive in terms of
effort and resources. A project's success majorly depends
on the precise identification of stakeholder's expectations and requirements for
the system they desired.
A possibility to do requirements elicitation is to mine available datasets e.g.
social media, requirement documents, and app stores reviews etc \levi{which documents?}. The latest trend for
identifying user requirements is to mine data obtained from platforms like Twitter,
Google Play Store, and Apple Store etc. These user reviews are not structured requirements
and contain information such as praises, dislikes or bug reports. Also, it contains extra information and noise that make manual requirement elicitation a difficult and challenging task.
Automated requirement elicitation is helpful in these cases and can
significantly reduce time, effort, and cost. This is mainly an ML
\emph{classification} task: give the set of information and identify it as a
requirement or not. Sometimes \emph{clustering} is also used for auxiliary
tasks.\\

Guzman \etal \cite{Guzman:2017} proposed the ALERTme approach for
classifying, grouping and ranking tweets during the software
evolution development. Many users shared their opinions about various software on Twitter. The huge amount of dataset made it hard to manually identify tweets that contained user requirements. The proposed methodology classified tweets as improvement requests or not, using \emph{Naive Bayes} algorithm. This was the first study of
its kind that was performed on software related tweets. 
The classifier was trained with following steps: 1) conversion of
pre-processed tweets into a VSM model, 2) train a classifier
on a set of manually annotated tweets, 3) predict the tweets categories using trained classifier. Furthermore, improvement
requests were considered for the grouping which helped to sort the
requests and summarize them accordingly. The summarization process contained highly ranked tweets based on parameters including likes, sentiments, and number of shares etc.\\

Williams \etal \cite{Williams:2017} performed a similar study on tweets in order
to classify them as user requirements. It used basic pre-processing techniques and applied VSM on data. For the learning
process, manually annotated (labelled) tweets were used and \emph{Naive Bayes} algorithm was applied for classification.\levi{annotated how, which ML
algorithm was used?}. The authors claimed with the help of results that
software tweets are neutral in nature, meaning sentiment analysis did not
influence the outcome of the ML algorithm. It showed improved results in comparison to \cite{Guzman:2017}.
The study used 4000 randomly selected tweets from ten different
softwares including Microsoft Visual Studio, Google Chrome, and Instagram etc\levi{software what?}.\\

Jiang \etal \cite{Jiang:2014} mined user reviews from 
app stores for discovering evolutionary requirements. It first extracted opinions about software features from reviews. For automated opinion identification, syntactic
relation based propagation approach was used that extracted targets and sentiment words
iteratively using known and extracted words. Afterwards, it applied k-mean clustering for opinion categorization.
\levi{explain what kind of clustering}. The proposed system also helped developers to decide requirements related to software revenue by considering economic factors. \levi{explain} It used two datasets of online reviews: one from the
Karplersky internet security 2011 software package (from Amazon) with 380
reviews; the other one  comprising 461 reviews for the  TuneIn Radio Pro V3.6
mobile app (from the app store).\\

Lange \etal\cite{Douglas:S2008} mapped the software requirement elicitation
process onto an existing military tool \emph{skiweb}. Skiweb was used to make decisions about what actions need to be taken on a military command.\levi{which decision?}. Different users posted and updated events and information using this tool. The goal of adding learning capability was to find additional information relevant to user posts. The proposed recommender system used supervised
\emph{Naive Bayes} algorithm to classify text documents in order to find related
requirements to the post \levi{related to what?}. Furthermore, it
\levi{which recommender system?} used topic modeling to identify the key
stakeholders and suggested them requirements for further analysis according to their interest\levi{which analysis?}. This study used an internal organizational
dataset Skiweb Data such as wiki, blogs etc.\\

Jha \etal \cite{Jha:2017} discovered user requirements by mining app store reviews. The requests were classified into three categories;
\emph{bugs}, \emph{features}, and \emph{junk}. The proposed methodology applied Naive Bayes and SVM. The distinction between types of sentences was identified by frame semantics \emph{explain frame semantics} instead of text classification methods. It generated frames for each review, rather than each word. Due to small number of features, a slower dimensional model was produced with enhanced prediction capabilities. It combined existing
datasets from past studies and reviews for iOS apps including CreditKarma,
Fitbit, and Gmail.

Maalej presented in \cite{Maalej} a study on how to classify app reviews as bug
reports, feature requests, user experiences, and ratings. It used
\emph{Naive Bayes} algorithm due to better results in comparison to other algorithms for classification. It also highlighted that binary
classifier performed better than multi
classifiers. It used meta model to enhance the classification performance e.g. ratings,
tense, and sentiment scores etc. A
dataset of 4400 manually annotated reviews from Google Play Store and the Apple App
Store were used for the study.\\

Herrera \etal \cite{Castro-Herrera:2009} built a semi-automated\levi{why
semi-automated?} recommender system to manage a large number of stakeholders participation in
the requirements elicitation and prioritization process. In this system, stakeholders could work collaboratively to transform their needs into sets
of articulated and prioritized requirements. It automatically
generated specialized topics for building forums for stakeholders collaboration and discussion. Stakeholders interests were depicted from their user profiles, that also helped to create recommendations according to the
interest of a community of similar stakeholders.\levi{is this the recommender
part?} For identifying topics, an unsupervised agglomerative clustering algorithm was
applied to unstructured data\levi{what's the difference?}. The proposed system analyzed online
datasets\levi{which datasets?} that were gathered from stakeholders in natural language. 
The evaluation dataset was a collection of 36 feature requests created by
graduate-level students for an Amazon-like student web-portal system.







\subsection{Requirements Specification and Analysis}

\subsubsection{Non-Functional}

Non-functional requirements may not be explicitly mentioned in a formal specification requirements documents even though, all systems have them~\cite{Slankas:2013}. Moreover, freeform documents like interview notes, meeting minutes and scattered requirements specifications include non-functional requirements which need to be detected and classified. In order to support analyst in the error-prone task of manually discovering and classifying NFRs machine learning can be useful. Automatic detection can be used to quickly and more effectively analyze large and complex documents for searching the NFRs\cite{Cleland-Huang2007}. This is a classification problem as from a set of requirements we want to decide a class membership. \\

One of the studies is by Slankas \etal~\cite{Slankas:2013} where they automatically identified and classified sentences in natural language from use agreements, install manuals, regulations, request for proposals, requirements specifications, and user manuals output into 14 different NFRs categories: Access Control (AC), Audit (AU), Availability (AV), Legal (LG), Look and Feel (LF), Maintenance (MT), Operational (OP), Privacy (PR), Recoverability (RC), Performance and Scalability (PS), Reliability (RL), Security (SC), Usability (US). Their two-step process: 1) parse natural language and turn sentences into graphs 2) classify sentences into categories led them into finding 20 keywords for each category of NFRs. \newline
Cleland-Huang \etal~\cite{Cleland-Huang2007} provided the same approach and used supervised classification for grouping non-functional requirements: availability, look-and-feel, legal, maintainability, operational, performance, scalability, security, and usability.

\subsubsection{Functional}

Software requirements specifications are usually stated in informal, imprecise and ambiguous natural language, thus analyzing them is a challenging task. However, for requirements reuse in Software Product Line analyzing is a vital task. Automatically extract structured information of functional requirements from Software Requirements Specifications and grouping them into different categories is a machine learning classification task\cite{7949577}. \\

Wang \etal~\cite{7949577} applied a combination of machine learning, natural language processing, and semantic analysis methods for automatically extract non-functional requirements and classify them into 10 different cases: Agentive, Action, Objective, Agent mode, Objmod, Locational, Temporal, Manner, Goal, Constraint. Their framework has four steps: corpus construction,  NLP  preprocessing, feature extraction and  EFRF (Extended Functional Requirements Frame) functional cases extraction. which for NLP processing they did tokenization,   lemmatization, part-of-speech tagging (POS  tagging) and dependency parsing. Ultimately, they showed that their trained model on E-commerce requirements dataset can be used to extract semantic information from the requirements of automaker systems.

\subsubsection{Functional and Non functional Requirements}

The success of system solely not depends on functional requirements. Just as functional requirements, it also significantly depends upon the adherence to non-functional requirements. In general, the primary focus is more for identification and specification of the FR.  NFRs usually identified and specified in the late development process that can increase the risks. FRs tend to be more straightforward e.g. store and retrieve transaction. On the other side, NFRs are complicated and challenging to implement e.g., making the design to meet NFRs or design test case for them. Different types of requirements analyzed in different way and it is useful to have separate division to look at one particular class. That is why it is necessary to distinguish between FR and NFR and categorize NFRs into subcategories. This distinction help to manage changes in requirements. The manual distribution is difficult and time consuming. Machine learning can be used for reducing the effort and categorizing the requirements based on the text segment analysis. This is ML classification task: give the set of requirements and identifying its category. \\

\etal~\cite{} Automatically classify the user review text into FR, NFR and others. It further classify the NFR into four categories as reliability, usability, portability, and performance. It used supervised machine learning algorithm (bagging) for training the classifier. The text was trimmed by stop words elimination, lemmatization, stemming, and sentences split. For augmenting the user review word2vec used. Word2Vec is actually a two layer neural network to process text for finding the word embedding. \etal~\cite {Deocadez:2017} performed semi-supervised classification techniques for automated classification of FR and NFR in user reviews from the app store. This study deals with two problems: 1) minimize annotate or label the big dataset of user reviews 2) classification of FR and NFR. First problem solved by using semi-supervised self-labeling algorithm. Self-labelling algorithms needs only small amount of dataset to get the comparable results with supervised techniques. For classification problem Naïve Bayes algorithm used. Features are obtained by applying standard text mining technique and additional attribute embellishment. For text mining technique following features used:  Inverse Document Frequency (IDF) Transform, Term Frequency (TF) Transform, Lowercase transformation, Minimum term frequency, Stemmer, Number of words. The second stage involved removing numbers, 2- letter words and other symbolic characters.\etal~\cite{Kurtanovic:2017}performed automated analysis of number of software requirements and performed classification on FR, NFR and its different categories using supervised machine learning algorithm(support vector machine). Additional dataset of user comments from Amazon used to overcome the data imbalance problem in NFR. For feature extraction text preprocessing techniques used such as removal of punctuations, removal of stop words, and lemmatization. 

\cite{Deocadez:2017}, \cite{Kurtanovic:2017}, \cite{Guzman:2017},
\cite{Abad:2017}, \cite{Dekhtyar:2017}, \cite{Rashwan:2012}, \cite{Lu:2017},
\cite{Hayes:2014}, \cite{Williams:2017}, \cite{Garzoli:2013},
\cite{Casamayor:2010}, \cite{Wang:2016}, \cite{Hussain:2012}, \cite{Jiang:2014},
\cite{Jha:2017}, \cite{Pinquie:2015}

\subsubsection{Security Requirements} 

Due to the orthogonal character of their impact on a system, \emph{security}
requirements are notoriously difficult to identify, objectify and
quantify~\cite{}. Also during requirement specification, it very often
happens that security requirements are masked by functional requirements
(but can be deduced from the context of the domain the system operates
in)~\cite{Riaz:2014}. Because of this, it often happens in practice that
security requirements are only marginally tackled during system
construction~\cite{}, paving the way to potentially catastrophic consequences.
Machine learning can be of use here by aiding in the identification of segments
of text that describe security requirements. This is a \emph{classification}
problem: given a text, identify which parts of it correspond to which
type of security issues.\\

Jindal\etal~\cite{Jildal:2016} automatically learn decision trees that can be
used to classify security requirements as \emph{authentication}, \emph{access
control}, \emph{encryption} or \emph{data integrity}. The \emph{features} used
are relevant terms found in the text. Such relevant terms are obtained by the
following sequence of actions: 1) removing stop words from the text; 2) stemming
the remaining words; and 3) ranking the stemmed words by their \emph{info-gain}
measure.

Riaz and her colleagues~\cite{Riaz:2014} use the k-nearest neighbors algoritm to
classify sentences in requirements documents as \emph{confidentiality},
\emph{integrity}, \emph{authentication}, \emph{availability},
\emph{accountability} or \emph{privacy} requirements. In order to find
adequate sentences and provide context to the classifier, the authors start by
finding a type for each sentence among the possibilities \emph{title}, \emph{list start}, \emph{list
element} or \emph{normal sentence}. For the classification the authors use a
modified version of the Levenshein distance based on the number of word transformations needed to go
from one term in one sentence to a term in another sentence. The classifier is
trained using requirements sentences from the healthcare domain that are
manually classified. A particularity of the approach is that each security
requirement type is associated to a template that helps in translating the
security requirements into functional requirements in order to ease during the
implementation of the final system. 


\cite{Knauss:2011} 

\subsubsection{Contextual Requirements} 


\subsection{Requirements Validation}

\subsubsection{Traceability}

Validation is to guarantee that requirements are reflecting stakeholders' needs,
confirm the quality of the system, consistency, and traceability.
In requirements traceability, the emphasis is on the ability to track the life of
requirements and their established links within other artifacts. However, the
main barrier assures traceability is the needed effort for building and
maintaining the links between those artifacts. That is why many research has
tried to apply machine learning and automated tools for facilitating the
establishment of links\cite{Gervasi:2011}.
Traceability tackled in the research mainly by the use of machine learning
classification and reinforcement learning methods.

Gervasi \etal~\cite{Gervasi:2011} investigate what can be learned from links that are already established. They build classifiers as a mean to develop models of tracing that can then be interpreted by humans to understand how requirement tracing is done in practice. Their purpose is to revise the existing models of hard-coded traceability tools such as VSM. They used a publicly-available dataset of requirements with traceability information, originally based on the CM-1 project by the NASA Metrics Data Program. Their approach has the following steps: applied preprocessing techniques and transferred requirements into a vector of features which from them derived set of classification cases by joining one high-level requirement and one low-level requirement and adding a classification of \textit{link} or \textit{nolink} based on whether that particular pair was a true link in the original dataset, or not. Finally, use the dataset to train and test two different classifiers from the WEKA collection, a Naive Bayesian classifier, and the J48 decision-tree classifier.

Sultanov \etal~\cite{Sultanov:2013} finds traceability candidates from
high-level to low-level requirements by the use of reinforcement
learning. They used textual high and low-level requirements documents as an input and try to
find the candidate traces. Their technique demonstrated statistically
significantly better results than the Information Retrieval technique.






\subsection{Requirements Management}

\subsubsection{Visualization}

Natural language requirement documents can be hard to comprehend and analyze.
Similarly, stakeholders have to review and understand requirements for large and
complex systems.  In these scenarios, basic information visualizations, like
charts and graphs have been used in requirements engineering.
These visualizations are usually applied to improve textual requirements with
summarization that combined large amounts of information into a single
representation for quick consumption by stakeholders\cite{Reddivari:2012}.
Machine learning is of great value in discovering visualized groups of large
numbers of requirements artifacts\levi{I think you mean ``machine learning
is useful in grouping requirements for visualization purposes''}.\levi{which
kind of machine learning taks? Clustering?}\\

ReCVisu (Requirements Clustering Visualization) tool is presented in Reddivari
\etal~\cite{Reddivari:2012} paper. ReCVisu, an exploration tool based on
quantitative visualizations helps requirements engineers understand the nature
of the requirements in a visual form. In ReCVisu, the dependence graph consists
of requirements artifacts as nodes and the textual similarities as edges. The
automatic grouping of requirements into clusters can help in areas such as
uncovering the requirements structure, navigating around the requirements space,
modularizing crosscutting concerns, and understanding requirements interactions
and evolution.\newline

Pinqui \etal~\cite{Pinqui:2015} recognize the enormous volume of requirements as
big data with which companies struggle to make strategic decisions early on.
Therefore, they built a complete visual framework to filter requirements from
stakeholders in a way that architects can make better insightful decisions. They
suggest training a multi-class SVM model from domain-specific (mechanics,
electronics, etc.) dictionaries and handbooks. Overall, the paper proposes a
framework to go from management-oriented requirements to architecture-oriented
requirements in which SVM is only applied in a small part of it. \newline

Software requirements are mostly stated in natural text notations such as user
stories which is making it hard for people to develop an accurate mental image
of the most relevant entities and relationships. Lucassen
\etal~\cite{Lucassen:2016} introduced an automated method for visualizing
requirements at different levels of granularity. Their visualization method from
user stories consists of 1) the generation of an overview which provides a
general context for understanding the dataset:
\begin{itemize}
\item Extract a set of relevant concepts from the user stories and their relationships 
\item Calculate the semantic similarity by using skip-gram implementation word2vec
\item Utilize Ward’s clustering algorithm to group all the concepts according to their similarity 
\item Identify the concept which is most similar to the collection of concepts in a cluster
\item Generate inter-cluster relationships matrix
\item Visualization Drawing
\end{itemize}
2) zooming in and out mechanisms and 3) filtering techniques to reduce the
complexity of the data presentation. Possible anticipated applications of this
visualization are: discovering missing relationships between clusters that may
result in further user stories, teaching system functionality by exploring
simplified, manageable chunks, and analyzing expected system changes after
introducing new sets of user stories.

\subsubsection{Structuring Documents} 

Requirements of the system are usually presented in natural language documents. These documents require to be properly structured for a better overall understanding of the requirements. For this purpose, the document should be organized with independent sections which each one contains conceptually connected requirements\cite{Ferrari:2013}. Moreover, technical review is a usual way to guarantee the quality in natural language specifications. However, extensive and comprehensive specifications make it problematic for reviewers to find defects, especially consistency or completeness ones. Therefore, use of machine learning algorithms can support reviewers with their work by automatically classifying and clustering the information that is spread over many sections of many documents \cite{Ott:2013}.


\cite{Ferrari:2013}

\cite{Rauf:2011}

\cite{Ott:2013}
 