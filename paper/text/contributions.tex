\section{Contributions}

\levi{summarize preprocessing steps and feature calculation}
\levi{disclaimer: sometimes features are presented sometimes not}
\levi{?} Stakeholders needs
(features)\levi{features in this case are just words} are initially preprocessed
by removing common (stop) terms, and computation of tf-id \levi{explain tf–idf,
maybe extract it to a common section, in general explain that features are
words and the instances are tweets/reviews/\ldots} for each term.

Hollis \etal \cite{Hollis2017} proposed an initial study to automate requirement
elicitation in Agile environment by providing the list of words and loosely
formatted list of requirements. The proposed methodology applied text mining
technique on recorded conversation of the stakeholder and developer
conversation. It was a short paper and did not provide details.\\

Dong \etal \cite{dong2010} also applied text mining on different form of
document and resources from internet for gathering requirement. The system
applied data preprocessing as word segmentation and stop words removal and build
up the VSM model. Kaiya \etal \cite{Kaiya:2010} proposed a tool to improve the
domain knowledge ontology for requirement elicitation by using web mining and
NLP technique. It helped to mine the general concepts to ontology for
requirements elicitation.


\subsection{Requirements Elicitation and Discovery}

The manual process of requirement elicitation is time-consuming and required
efforts, and sources. The project success majorly depends upon the precise
identification of stakeholder expectations and requirements for their desired
system. Poor elicitation can lead to over budget and time issues. Requirements
elicitation can be done by mining the available dataset e.g., social media,
documents, and gathering and analyzing information available on the internet,
publically or privately available data. In short, gathering data from different
means and analyze them for requirements elicitation.  The latest trend for
identification of user requirement is mining social media like twitter, google
store, and app store data. These user reviews are not structured requirement and
contained other information too such as praise, dislikes, bug report etc.  The
unstructured, extra information and noise presence in the above-described
sources makes the manual process difficult and challenging. The automated
analysis of requirement elicitation is helpful and can significantly reduce
time, effort, and cost. This is ML classification task: give the set of
information and identifying it as a requirement or not.
\\

\subsubsection{Internal vs External}
Guzman \etal \cite{Guzman:2017} proposed ALERTme approach for classifying,
grouping and ranking tweets in software evolution process. For identification of
evolutionary requirement supervised classification performed on tweets as
improvement request or not using Naïve Bayes algorithm. It was the first kind of
this study that finds user requirements from tweets for software. Williams \etal
 \cite{Williams:2017} performed a similar study on
tweets for classifying as user requirement or not. The study gave state of the
art results using Naïve Bayes and VSM\levi{introduce VSM?} algorithm. Different
techniques were implemented for feature extraction that includes stopword removing, sentimental
analysis, stemming\levi{bring to machine learning section}, and Bag of words.
For learning process maually annotated tweets were used. However, results showed
that software tweets are neutral in nature and sentiment analysis did not
influence ML algorithm.\\

Research on mining user reviews in mobile application (app) stores has
remarkably advanced in the past few years. Jiang \etal \cite{Jiang:2014}
presented an optimized method for discovering the evolutionary requirements for
developers. This method clustered opinion expressions in form of a macro network
topology, and combine polarized sentimental analysis with the economic factors
to decide evolutionary requirements. The dataset used POS tagger and parser with
additional defined rules for feature extraction. Douglas S. ~etal
\cite{Douglas:S2008} mapped software requirement elicitation process to an
existing military tool skiweb that used for making the decision. The proposed
methodology used supervised Naive Bayes to classify text document to find
related requirements. Furthermore, recommendation system used topic modeling to
identify the key stakeholder for which that requirement is important and allow
for the analysis. Jha \etal \cite{Jha:2017} Classified application store reviews
and discover user requirement as feature request. The data was classified into
three categories bugs, features, and junk. For classification, SVM and Naïve
Bayes algorithm was used. The study used frame semantic analysis that genralize
more abstract contexts. This technique produced slower dimensional model with
smaller number of features, that help to enhance the prediction
capabilities.Herrera\etal \cite{Castro-Herrera:2009} proposed a tool to scale up
the support for large-scale systems requirements processes in comparison to
traditional requirements elicitation process. The large oftware systems are
complex due to this it is hard to manage stakeholders and identify the relevant
stakeholders for further discussions.
The proposed recommender framework used feature requests of different projects
in form of natural language statements.
An unsupervised clustering techniques used to identify cohesive themes out of
these features. Then tool constructed user profiles according to the interests
of the stakeholders in each of these themes for recommender system. Stakeholders
needs (features) are initially preprocessed by removing common (stop) terms, and
computation of tf-id for each term. \\



\subsubsection{Text mining} 
 Herrera \etal \cite{Castro-Herrera:2009} provide semi-automatically manage broad stakeholder participation in the requirements elicitation and prioritization process. 
The proposed system analyzed datset i.e. gathered from stakeholders and automatically generated highly specialized topical forums i.e. themes identification, and assign it to stakeholders accordingly.
On these forums stakeholder can work collaboratively to transform statements of need into sets of articulated and prioritized requirements. For themes identification unsupervised clustering
techniques applied to massive amounts of unstructured or semi-structured data. The dataset was  small collection of featur requests created by 36 graduate level students for an Amazon-like student web-portal system. Hollis \etal \cite{Hollis2017} proposed an initial study to automate requirement elicitation in Agile environment by providing the list of
words and loosely formatted list of requirements. The proposed methodology
applied text mining technique on recorded conversation of the stakeholder and
developer conversation. Dong ~etal \cite{dong2010} also applied text mining on
different form of document and resources from internet for gathering
requirement. The system applied data preprocessing as word segmentation and stop
words removal and build up the VSM model. Kaiya \etal \cite {Kaiya:2010} proposed a
tool to improve the domain knowledge ontology for requirement elicitation by using web mining and NLP
technique.It helped to mine the general
concepts to ontology for requirements elicitation.
  \levi{not clear what the \ML
contribution is here. Model, features and instances are missing.} 



\subsection{Analysis of Requirements}

\subsubsection{Non-Functional}

\cite{Slankas:2013}

\subsubsection{Functional}

\cite{Deocadez:2017}, \cite{Kurtanovic:2017}, \cite{Guzman:2017},
\cite{Abad:2017}, \cite{Dekhtyar:2017}, \cite{Rashwan:2012}, \cite{Lu:2017},
\cite{Hayes:2014}, \cite{Williams:2017}, \cite{Garzoli:2013},
\cite{Casamayor:2010}, \cite{Wang:2016}, \cite{Hussain:2012}, \cite{Jiang:2014},
\cite{Jha:2017}, \cite{Pinquie:2015}

\subsubsection{Security Requirements} 

\cite{Jildal:2016}, \cite{Riaz:2014}, \cite{Knauss:2014} 

\subsubsection{Contextual Requirements} 


\subsection{Requirements Validation}

\subsubsection{Traceability}

Validation is to guarantee that requirements are reflecting stakeholders' needs,
confirm the quality of the system, consistency, and traceability.
In requirements traceability, the emphasis is on the ability to track the life of
requirements and their established links within other artifacts. However, the
main barrier assures traceability is the needed effort for building and
maintaining the links between those artifacts. That is why many research has
tried to apply machine learning and automated tools for facilitating the
establishment of links\cite{Gervasi:2011}.
Traceability tackled in the research mainly by the use of machine learning
classification and reinforcement learning methods.\\

Gervasi \etal~\cite{Gervasi:2011} investigate what can be learned from links that are already established. They build classifiers as a mean to develop models of tracing that can then be interpreted by humans to understand how requirement tracing is done in practice. Their purpose is to revise the existing models of hard-coded traceability tools such as VSM. They used a publicly-available dataset of requirements with traceability information, originally based on the CM-1 project by the NASA Metrics Data Program. Their approach has the following steps: 1) applied preprocessing techniques and transferred requirements into a vector of features 2) From these vectors derive set of classification cases by joining one high-level requirement and one low-level requirement and adding a classification of \textit{link} or \textit{nolink} based on whether that particular pair was a true link in the original dataset, or not 3) finally, use the dataset to train and test two different classifiers from the WEKA collection, a Naive Bayesian classifier, and the J48 decision-tree classifier.\newline

Sultanov \etal~\cite{Sultanov:2013} finds traceability candidates from
high-level to low-level requirements by the use of reinforcement
learning. They used textual high and low-level requirements documents as an input and try to
find the candidate traces. Their technique demonstrated statistically
significantly better results than the Information Retrieval technique.






\subsection{Requirements Management}

\subsubsection{Visualization}

Natural language requirement documents can be hard to comprehend and analyze.
Similarly, stakeholders have to review and understand requirements for large and
complex systems.  In these scenarios, basic information visualizations, like
charts and graphs have been used in requirements engineering.
These visualizations are usually applied to improve textual requirements with
summarization that combined large amounts of information into a single
representation for quick consumption by stakeholders\cite{Reddivari:2012}.
Machine learning is of great value in discovering visualized groups of large
numbers of requirements artifacts\levi{I think you mean ``machine learning
is useful in grouping requirements for visualization purposes''}.\levi{which
kind of machine learning taks? Clustering?}\\

ReCVisu (Requirements Clustering Visualization) tool is presented in Reddivari
\etal~\cite{Reddivari:2012} paper. ReCVisu, an exploration tool based on
quantitative visualizations helps requirements engineers understand the nature
of the requirements in a visual form. In ReCVisu, the dependence graph consists
of requirements artifacts as nodes and the textual similarities as edges. The
automatic grouping of requirements into clusters can help in areas such as
uncovering the requirements structure, navigating around the requirements space,
modularizing crosscutting concerns, and understanding requirements interactions
and evolution.\newline

Pinqui \etal~\cite{Pinqui:2015} recognize the enormous volume of requirements as
big data with which companies struggle to make strategic decisions early on.
Therefore, they built a complete visual framework to filter requirements from
stakeholders in a way that architects can make better insightful decisions. They
suggest training a multi-class SVM model from domain-specific (mechanics,
electronics, etc.) dictionaries and handbooks. Overall, the paper proposes a
framework to go from management-oriented requirements to architecture-oriented
requirements in which SVM is only applied in a small part of it. \newline

Software requirements are mostly stated in natural text notations such as user
stories which is making it hard for people to develop an accurate mental image
of the most relevant entities and relationships. Lucassen
\etal~\cite{Lucassen:2016} introduced an automated method for visualizing
requirements at different levels of granularity. Their visualization method from
user stories consists of 1) the generation of an overview which provides a
general context for understanding the dataset:
\begin{itemize}
\item Extract a set of relevant concepts from the user stories and their relationships 
\item Calculate the semantic similarity by using skip-gram implementation word2vec
\item Utilize Ward’s clustering algorithm to group all the concepts according to their similarity 
\item Identify the concept which is most similar to the collection of concepts in a cluster
\item Generate inter-cluster relationships matrix
\item Visualization Drawing
\end{itemize}
2) zooming in and out mechanisms and 3) filtering techniques to reduce the
complexity of the data presentation. Possible anticipated applications of this
visualization are: discovering missing relationships between clusters that may
result in further user stories, teaching system functionality by exploring
simplified, manageable chunks, and analyzing expected system changes after
introducing new sets of user stories.

\subsubsection{Structuring Documents} 

Requirements of the system are usually presented in natural language documents. These documents require to be properly structured for a better overall understanding of the requirements. For this purpose, the document should be organized with independent sections which each one contains conceptually connected requirements\cite{Ferrari:2013}. Moreover, technical review is a usual way to guarantee the quality in natural language specifications. However, extensive and comprehensive specifications make it problematic for reviewers to find defects, especially consistency or completeness ones. Therefore, use of machine learning algorithms can support reviewers with their work by automatically classifying and clustering the information that is spread over many sections of many documents \cite{Ott:2013}.\\

Duan \etal~\cite{Duan:2007} used hierarchical automated clustering technique for detecting cross-cutting concerns as it is beneficial for the process of requirements analysis and architectural design. The reported experiments in this paper were supported by two tool sets, Poirot, a web-based tool designed to generate traces between various software engineering artifacts which was applied to compute similarity scores between requirements and a developed prototype tool, capable of reading structured requirements specification and generated similarity scores and then clustering requirements. \newline

Requirements engineering process results are usually documented in the natural language specifications. In most cases, these documents not only contain requirements but also some additional information such as explanations, summaries, and figures. As it is important to differentiate between relevant requirements and other auxiliary content it is often the case that requirements engineers manually label each element of the specification document. Winkler \etal~\cite{Winkler:2016} applied convolutional neural networks to automatically classify content elements of a natural language requirements specification as “requirement” or “information”.\parisa{requirements elicitation or management?} Their approach increases the quality of requirements specifications as it distinguishes important content for activities like test and etc. For converting natural language into a vector representation word2vec method is used. A set of 10000 content elements extracted from 89 requirements specifications of an industry partner used for training the network through the use of Tensorflow library using stochastic gradient descent. \newline 

For having a better understanding the natural language requirements specification documents should be properly structured. two quality characteristics of such a document are requirements relatedness which is each requirement is conceptually connected with the requirements in the same section and sections independence which is each section is conceptually separated from the others. based on Ferrari \etal~\cite{Ferrari:2013} automatically recognizing the sections in the document that need requirements relatedness and sections independence may help enhance the document structure. The authors defined a novel algorithm named Sliding Head-Tail Component (S-HTC) for clustering the requirements according to relatedness (the algorithm is based on known distance - Jaccard similarity metric, Levenshtein distance and, the convex combination between $\sigma$jac and $\sigma$lev). The algorithm groups together similar requirements that are contiguous in the requirements document. The effectiveness of the algorithm was evaluated with a test on requirements standard of a railway domain (583 requirements). \newline

Based on Rauf \etal~\cite{Rauf:2011} software specification documents usually contain instances of logical structures, such as business rules, use cases, and functional requirements. Automated identification and extraction of these instances will benefit requirements management features, like automated traceability, template conformance checking, and guided editing. The authors planned a framework that gets requirements documents as an input and tries to develop a template for the general structure of it by specifying logical structures in terms of their content, textual rendering, and variability and then the extracting the instances of such structures from rich-text documents. \newline

Ott \etal~\cite{Ott:2013} automatically classified and extracted requirements with related information which are spread over many sections over many documents by the use of Multinomial Naive Bayes and Support Vector Machines classification algorithms as it will be helpful for reviewers with their work. As their input, they have used two German automotive specifications (Mercedes-Benz) which describe the functional and non-functional requirements of a Doors Closure Module (DCU). A specification and its referenced documents often sum up to 3,000 pages at Mercedes-Benz. Their method collects requirements of related information into classes, which they call topic landscape and later they built a tool, ReCaRe(Review with Categorized Requirements) which is the realization of the topic landscape based on eclipse with a data connection to IBM Rational DOORS. 





 