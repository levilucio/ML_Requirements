\subsection{Requirements Specification and Analysis}

\subsubsection{Non-Functional}

\cite{Slankas:2013}

\subsubsection{Functional}

\cite{Deocadez:2017}, \cite{Kurtanovic:2017}, \cite{Guzman:2017},
\cite{Abad:2017}, \cite{Dekhtyar:2017}, \cite{Rashwan:2012}, \cite{Lu:2017},
\cite{Hayes:2014}, \cite{Williams:2017}, \cite{Garzoli:2013},
\cite{Casamayor:2010}, \cite{Wang:2016}, \cite{Hussain:2012}, \cite{Jiang:2014},
\cite{Jha:2017}, \cite{Pinquie:2015}

\subsubsection{Security Requirements} 

Due to the orthogonal character of their impact on a system, \emph{security}
requirements are notoriously difficult to identify, objectify and
quantify~\cite{}. Also during requirement specification, it very often
happens that security requirements are masked by functional requirements
(but can be deduced from the context of the domain the system operates
in)~\cite{Riaz:2014}. Because of this, it often happens in practice that
security requirements are only marginally tackled during system
construction~\cite{}, paving the way to potentially catastrophic consequences.
Machine learning can be of use here by aiding in the identification of segments
of text that describe security requirements. This is a \emph{classification}
problem: given a text, identify which parts of it correspond to which
type of security issues.\\

Jindal\etal~\cite{Jildal:2016} automatically learn decision trees that can be
used to classify security requirements as \emph{authentication}, \emph{access
control}, \emph{encryption} or \emph{data integrity}. The \emph{features} used
are relevant terms found in the text. Such relevant terms are obtained by the
following sequence of actions: 1) removing stop words from the text; 2) stemming
the remaining words; and 3) ranking the stemmed words by their \emph{info-gain}
measure.

\cite{Jildal:2016}, \cite{Riaz:2014}, \cite{Knauss:2011} 

\subsubsection{Contextual Requirements} 

