\section{Threats to Validity}
\label{sec:threats_to_validity}

The validity of our study might be affected by the coverage of the search
results, bias on the selection of studies, and inaccuracy of data extraction.

\emph{Study Coverage}: The study we present here is partial, meaning
relevant work could be missing due to inadequate search strings or
the list of databases not being complete. The data preparation was based as
much as possible on a systematic method, which resulted on a map of the read
articles and their main features as relevant to our study.

\emph{Study Selection Bias}: we understand that the assessment might be biased
by the interests of the involved researchers. As such, the themes that we discuss
in this article may be influenced by the preferences of the involved
researchers. To mitigate for this threat, a set of include and exclude criteria
was predefined and researchers assessed the title and abstract of the papers to
steer the research. Many of the papers retrieved by our queries apply \NLP
to requirements engineering but involve no learning
(e.g.~\cite{Xiao:2012},~\cite{Deeptimahanti:2011},~\cite{ChengHeLiangLi:2010},
to cite a few). We have explicitly excluded such papers from our survey:
although \NLP tools do sometimes include \ML algorithms, their functionality is
used in a black-box manner by \RE researchers and as such they were not taken
into consideration.

\emph{Inaccuracy of Data Extraction}: given the data extraction process might be
biased by researcher interest, the selection of data items was strictly driven
by the research questions. Moreover, reading assignments were marked by the
researchers depending on their confidence level. Low-confidence assignments were
discussed between the authors until a consensus was reached.