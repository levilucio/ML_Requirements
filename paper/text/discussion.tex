\section{Discussion}

Our survey work implicitly points to a number of trends that we will
concretize and summarize in this section. Note that while the pointers we
provide here informed by the literature review we conducted, this survey is not
fully systematic (as described in section~\ref{sec:threats_to_validity}) which
means our conclusions may be revised and/or extended by future surveys of the
domain.

Firstly, it is obvious from our survey that \NLP techniques are heavily used
throughout a majority of the research tacking the application of \ML to
\RE. This is not surprising and even intuitive. \RE is the area of software
engineering where natural language is employed more ubiquitously, as \RE
techniques and tools play the role of interface between stakeholders such as
clients, certification entities, architects or developers. Although many
attempts have been done to bring formality to requirements
engineering~\cite{Teufl17,LucioRCA16}, the \emph{de facto} language between
technical and non-technical stakeholds for real-world projects continues being natural language, and in particular English.
The IBM Rational \DOORS family~\cite{doors} of tools is an example
of a natural-language based tool for requirement engineering that has become the
reference in many domains. In the techniques we have observed, \NLP is heavily
used for the preprocessing stages of natural language in order to bring the data to a format
that can be consumed by a learning algorithm (see
section~\ref{sec:preprocessing}).

Many of the papers retrieved by our queries apply \NLP to requirements
engineering but involve no learning (e.g.~\cite{Xiao:2012},~\cite{Deeptimahanti:2011},~\cite{ChengHeLiangLi:2010}, to
cite a few). We have explicitly excluded such papers from our survey: although
\NLP tools do sometimes include \ML algorithms, their functionality is used in a black-box
manner by \RE reserchers and as such cannot be taken into consideration by our
work.\levi{maybe this last paragraph should go to the protocol part}

\begin{itemize}
  \item Not better than SOA, comparations with humans, comparison with non-ML
  techniques
  \item There are a number of datasets available
  \item Clustering is on par with classification
  \item There are more contributions on Elicitation and Analysis
\end{itemize}

\newcommand\nocell[1]{\multicolumn{#1}{c|}{}}
\begin{table*}%
	\begin{scriptsize}
\begin{center}
	\begin{tabular}{|c|l|l|l|l|l|}
		\cline{2-6}
		\nocell{1} & \textbf{Themes} & \textbf{Contributions} & \textbf{ML Task} &
		\textbf{\ML Model Types} & \textbf{Datasets
		Used}\\
		\cline{2-6}
		\cline{1-6} 
		\multirow{3}{*}{\rotatebox[origin=c]{90}{\textbf{E}}}
			& External  & \cite{Guzman:2017}($\circ$)  \cite{Williams:2017}(+)
			\cite{Jiang:2014}(+) & Classification & (Multinomial) Na\"ive Bayes & Online
			reviews for KIS 2011 (from Amazon) \\
			&& \cite{Douglas:S2008}($\circ$) \cite{Jha:2017}(+)
			\cite{Castro-Herrera:2009}($\circ$) & Clustering & Support Vector Machines &
			Skiweb data \\
			\cdashline{2-6}[.4pt/1pt]
		\hline
		\multirow{3}{*}{\rotatebox[origin=c]{90}{\textbf{S}}} 
			& Non-Functional & ~\cite{Slankas:2013}($\circ$) ~\cite{Cleland-Huang2007}($\circ$)& Classification & k-Nearest Neighbors & Open Source PROMISE Datase\footnote{http://promise.site.uottawa.ca/SERepository/datasets-page.html}\\
			\cdashline{2-6}[.4pt/1pt]
			& Functional & ~\cite{7949577}($\circ$) & Classification & Bi-Directional
			Long Short-Term Memory &\\
			&&&&Conditional Random Field Network & - \\
			\cdashline{2-6}[.4pt/1pt]
			& Functional \& & \cite{Lu:2017}($\circ$) \cite{Deocadez:2017}($\circ$) \cite
			{Kurtanovic:2017}($\circ$) & Classification  &  Bagging,  Na\"ive Bayes, SVM & Open Source PROMISE Dataset \\
			& Non-Functional & \cite{Abad:2017}($\circ$) \cite {Garzoli:2013}($\circ$)
			~\cite{Wieloch:2013}($\circ$)  & & & app-store reviews \\
			\cdashline{2-6}[.4pt/1pt]
			& Prioritization & \cite{S.Dhingra}($\circ$) \cite{PAvesani}(+)
			\cite{7320432}($\circ$) & Classification & Case Based Ranking \ J.48 DT &  -
			\\
			\cdashline{2-6}[.4pt/1pt] 
			& Security & \cite{Jildal:2016}($\circ$) \cite{Riaz:2014}($\circ$) &
			Classification & Decision-Tree &\\&&&& k-Nearest Neighbors & - \\
			\cdashline{2-6}[.4pt/1pt]
		\hline
		\multirow{3}{*}{\rotatebox[origin=c]{90}{\textbf{V}}} 
			& Traceability & ~\cite{Gervasi:2011}($\circ$) ~\cite{Sultanov:2013}($\circ$) & Classification &  Na\"ive Bayes / J48 Decision-Tree & Open Source CM-1 NASA project\footnote{http://promise.site.uottawa.ca/SERepository/datasets/cm1.arff}\\
 		    & & & Reinforcement Learning & & Open Source Pine
 		    Dataset\footnote{http://www.washington.edu/pine/} \\
			\cdashline{2-6}[.4pt/1pt]
		\hline
		\multirow{3}{*}{\rotatebox[origin=c]{90}{\textbf{M}}}
			& Visualization & ~\cite{Reddivari:2012}($\circ$) ~\cite{Pinqui:2015}($\circ$) ~\cite{Lucassen:2016}($\circ$) & Classification & Support Vector Machines & \\
			& & & Clustering & Ward's method & - \\

			\cdashline{2-6}[.4pt/1pt]
			& Structuring & ~\cite{Duan:2007}($\circ$)
			~\cite{Winkler:2016}($\circ$)~\cite{Rauf:2011}($\circ$) & Classification & Multinomial  Na\"ive Bayes & International Union of Railways\\
			& & ~\cite{Ferrari:2013}($\circ$) ~\cite{Ott:2013}($\circ$) & Clustering &
			Support Vector Machines & (EIRENE
			Functional Requirements Specification\footnote{https://www.transportstyrelsen.se/globalassets/global/jarnvag/vagledning/godkannande/eirene($\_$)frsv7.pdf})\\
			& & & &  Convolutional Neural Networks & Mercedes-Benz car development\\
			&&&& Sliding Head-Tail Component Clustering & \\ 
			&&&& Hierarchical Clustering &\\
			\cdashline{2-6}[.4pt/1pt]
		\hline
	\end{tabular}
\end{center}
	\center{Legend: (+) improves the state of the art; (-) comparable to or worse
	than state of the art; ($\circ$) no information on how the approach
	relates to the state of the art}
	\end{scriptsize}
	\caption{Contributions and \ML tasks related to each theme within each \RE
	approach.}
	\label{tab:summary}
\end{table*}


\levi{internal vs external}