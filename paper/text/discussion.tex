\section{Discussion}

Our survey work implicitly points to a number of trends that we will
concretize and summarize in this section. Note that while the pointers we
provide here informed by the literature review we conducted, this survey is not
fully systematic (as described in section~\ref{sec:threats_to_validity}) which
means our conclusions may be revised and/or extended by future surveys of the
domain.

Firstly, it is obvious from our survey that \NLP techniques are heavily used
throughout a majority of the research tacking the application of \ML to
\RE. This is not surprising and even intuitive. \RE is the area of software
engineering where natural language is employed more ubiquitously, as \RE
techniques and tools play the role of interface between stakeholders such as
clients, certification entities, architects or developers. Although many
attempts have been done to bring formality to requirements
engineering~\cite{Teufl17,LucioRCA16}, the \emph{de facto} language between
technical and non-technical stakeholds for real-world projects continues being natural language, and in particular English.
The IBM Rational \DOORS family~\cite{doors} of tools is an example
of a natural-language based tool for requirement engineering that has become the
reference in many domains. In the techniques we have observed, \NLP is heavily
used for the preprocessing stages of natural language in order to bring the data to a format
that can be consumed by a learning algorithm (see
section~\ref{sec:preprocessing}). 



\begin{itemize}
  \item Not better than SOA, comparations with humans, comparison with non-ML
  techniques
  \item There are a number of datasets available
  \item Clustering is on par with classification
  \item There are more contributions on Elicitation and Analysis
\end{itemize}

\newcommand\nocell[1]{\multicolumn{#1}{c|}{}}
\begin{table*}%
	\begin{scriptsize}
\begin{center}
	\begin{tabular}{|c|l|l|l|l|l|}
		\cline{2-6}
		\nocell{1} & \textbf{Themes} & \textbf{Contributions} & \textbf{ML Task} &
		\textbf{\ML Model Types} & \textbf{Datasets
		Used}\\
		\cline{2-6}
		\cline{1-6} 
		\multirow{3}{*}{\rotatebox[origin=c]{90}{\textbf{E}}}
			& External  & \cite{Guzman:2017}($\circ$)  \cite{Williams:2017}(+)
			\cite{Jiang:2014}(+) \cite{Douglas:S2008}($\circ$) \cite{Jha:2017}(+) 
			\cite{Castro-Herrera:2009}($\circ$) & Classification / Clustering &
			(Multinomial) Na\"ive Bayes; & Online reviews for KIS 2011 (from Amazon) and TuneIn 3.6 (from App Store)
			\\&&&& Support Vector Machines & Skiweb data\\
			\cdashline{2-6}[.4pt/1pt]
		\hline
		\multirow{3}{*}{\rotatebox[origin=c]{90}{\textbf{A}}} 
			& Non-Functional & ~\cite{Slankas:2013}($\circ$) ~\cite{Cleland-Huang2007}($\circ$)& Classification & k-Nearest Neighbors & Open Source PROMISE Datase\footnote{http://promise.site.uottawa.ca/SERepository/datasets-page.html}\\ &&&&& Siemens Logistics and Automotive Organization requirement documents \\
			\cdashline{2-6}[.4pt/1pt]
			& Functional & ~\cite{7949577}($\circ$) & Classification & Bi-Directional Long Short-Term Memory\\&&&&Conditional Random Field Network & E-commerce Software Specification Documents \\
			\cdashline{2-6}[.4pt/1pt]
			& Functional \& Non-Functional & \cite{Lu:2017}($\circ$) \cite{Deocadez:2017}($\circ$) \cite {Kurtanovic:2017}($\circ$) \cite{Abad:2017}($\circ$) \cite {Garzoli:2013}($\circ$) ~\cite{Wieloch:2013}($\circ$)  & Classification  &  Bagging, Naive Bayes, SVM & Open Source PROMISE Datase, appstore reviews, internal dataset \\
			\cdashline{2-6}[.4pt/1pt]
			& Prioritization & \cite{S.Dhingra}($\circ$) \cite{PAvesani}(+) \cite{7320432}($\circ$) & Classification & Case Based Ranking \ J.48 DT & Software Requirements Documents in Natural Language\\
			\cdashline{2-6}[.4pt/1pt] 
			& Security & & & & \\
			\cdashline{2-6}[.4pt/1pt]
			& Contextuals & & & & \\
		\hline
		\multirow{3}{*}{\rotatebox[origin=c]{90}{\textbf{V}}} 
			& Traceability & ~\cite{Gervasi:2011}($\circ$) ~\cite{Sultanov:2013}($\circ$) & Classification / Reinforcement Learning & Naive Bayesian / J48 Decision-Tree & Open Source CM-1 NASA project\footnote{http://promise.site.uottawa.ca/SERepository/datasets/cm1.arff}\\ &&&&& Open Source Pine Dataset\footnote{http://www.washington.edu/pine/} \\
			\cdashline{2-6}[.4pt/1pt]
		\hline
		\multirow{3}{*}{\rotatebox[origin=c]{90}{\textbf{M}}}
			& Visualization & ~\cite{Reddivari:2012}($\circ$) ~\cite{Pinqui:2015}($\circ$) ~\cite{Lucassen:2016}($\circ$) & Classification / Clustering & Support Vector Machines / Ward's method& - \\
			\cdashline{2-6}[.4pt/1pt]
			& Structuring & ~\cite{Duan:2007}($\circ$) ~\cite{Winkler:2016}($\circ$) ~\cite{Rauf:2011}($\circ$) \\ && ~\cite{Ferrari:2013}($\circ$) ~\cite{Ott:2013}($\circ$) & Classification / Clustering & Multinomial Naive Bayes / Support Vector Machines \\ &&&& Convolutional Neural Networks \\ &&&& Sliding Head-Tail
Component Clustering \\ &&&& Hierarchical Clustering & International Union of
Railways: EIRENE Functional Requirements Specification\footnote{https://www.transportstyrelsen.se/globalassets/global/jarnvag/vagledning/godkannande/eirene($\_$)frsv7.pdf}\\ &&&&& Mercedes-Benz passenger car development Specifications\footnote{http://wwwbroy.in.tum.de/lehre/vorlesungen/ase/ss05/iese-002($\_$)02.pdf} \\
			\cdashline{2-6}[.4pt/1pt]
		\hline
	\end{tabular}
\end{center}
(+) improves the state of the art; (-) comparable to or worse
	than state of the art; ($\circ$) no information on how the approach
	relates to the state of the art\\\textbf{E}: Elicitation, \textbf{A}: Analysis, \textbf{V}: Verification
	\textbf{M}: Management
	\end{scriptsize}
	\caption{Contributions and \ML tasks related to each theme within each \RE
	approach.}
	\label{tab:summary}
\end{table*}


\levi{internal vs external}