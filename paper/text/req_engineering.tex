\subsection{Requirements Engineering}

Software systems are developed over millions of lines of code, number of modules
and documents. The primary goal of the software system is to satisfy users by
developing the software that can meet their needs and expectations. This goal is
achievable by applying different methodologies and engineering techniques. One
of the key factors here is to understand and identify the needs of users, also
known as, software requirements. Software requirements engineering is the
process that helps to determine the requirements in a systematic way to know what
functionalities the targeted system should have in order to fulfil user's
needs.

%Formally RE is defined as \cite{Zave:1997:CRE:267580.267581}:
%\begin{quote}
%``Requirements engineering is the branch of software engineering concerned with the real-world
%goals for, functions of, and constraints on software systems. It is also concerned with the
%relationship of these factors to precise specifications of software behavior, and to their evolution
%over time and across software families."
%\end{quote} 

Software requirements play a key role in the success of a project. In the USA, a
survey was conducted over 8380 projects by 350 companies to know the project
failure rates. The report's results \cite{StandishReport} showed only 16.2\%
projects were completed successfully and one-half (52.7\%) of projects met with challenges and
were completed with partial functionalities, time delays and over budget. Almost
31\% of the projects were never completed. The main cause identified by
executive managers was poor requirements engineering. The main culprits were the lack of
user involvement (13\%), incompleteness of requirements (12\%), changing
requirements (11\%), unrealistic expectations (6\%), and unclear objectives
(5\%). 

Software requirement engineering has mainly four phases; \emph{requirement
elicitation}, \emph{requirement analysis}, \emph{requirement documentation} and
\emph{requirement verification} \cite {Kotonya:1998}. Requirement elicitation
\cite{Coughlan:2002, Zowghi2005} helps to understand the
stakeholders needs, e.g. what features he/she wants in the software.
Requirement elicitation techniques are mostly derived by the social sciences,
organizational theory, knowledge engineering and practical experience. For
requirements elicitation, different techniques exist in the literature that
include interviews, questionnaires or ethnography.
Requirement analysis \cite{Nuseibeh:2000} phase emphasize on software requirements analysis 
to check for conflicts and consistency of requirements. It also makes sure
that the requirements are clear and complete.
Furthermore, the agreed requirements are documented in the documentation and verification phase. This documentation has a clear and precise
definition of the system functionalities that acts as an agreement between
stakeholders and developers. These requirements are documented usually as diagrams, mathematically
formulae or in natural language. These documents are used and iterated upon
until the end of the project.

System requirements are classified into business requirements, user
requirements, functional requirements (FR) and non-functional requirements (NFR). Functional
requirements are the system requirements that include the main features and
characteristics of the desired system. Non-functional requirements are the
system properties and constraints \cite{Davis:1993, Glinz}. NFRs set the
criteria for judging the operation of the system e.g.
performance, availability or reliability. Business requirements are specified to
address business objectives, vision, and goals. They are defined at a high level
to keep the knowledge from organization or company side for designing the products.
User requirements are the wish list for the system from users\levi{I don't
think the intersection between FRs, NFRs, BRs and URs is empty}.
User requirements are valuable for ensuring that system performs similarly as users wanted it to do.
