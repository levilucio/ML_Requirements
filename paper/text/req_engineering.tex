\subsection{Requirements Engineering}

Software systems are developed over millions of lines of code, software modules
and documents. The primary goal of a software system is to satisfy its users by
proposing functionalities that can meet their needs and expectations. This goal
is achieved by applying different methodologies and engineering techniques. One
of the key factors to satisfy this goal is to understand and identify the needs
of users through software requirements engineering. Software requirements
engineering is the process that helps in identifying software requirements in a
systematic manner in order to understand what functionalities the targeted
system should have in fulfil the users' needs.\par Software requirements play a
key role in the success of a project. In the USA, a survey was conducted over
8380 projects by 350 companies to understand project failure rates. The report's
results \cite{StandishReport} showed that only 16.2\% projects were completed
successfully while one-half (52.7\%) of the considered projects met with
challenges and were only partiallly compteted , with time delays and over
budget. Almost 31\% of the projects were never completed. The main cause for
such failures as identified by executive managers was poor requirements
engineering. In detail, main culprits were lack of user involvement (13\%),
incompleteness of requirements (12\%), changing requirements (11\%), unrealistic
expectations (6\%) and unclear objectives(5\%).

Software requirement engineering has traditionally four phases;
\emph{requirement elicitation}, \emph{requirement analysis}, \emph{requirement documentation} and
\emph{requirement verification} \cite {Kotonya:1998}. Requirement elicitation
\cite{Coughlan:2002, Zowghi2005} helps to understand the
stakeholders needs, e.g. what features he/she wants in the software.
Requirement elicitation techniques are mostly derived from the social sciences,
organizational theory, knowledge engineering and practical experience. For
requirements elicitation, different techniques exist in the literature such as
interviews, questionnaires or ethnography.
The requirements analysis \cite{Nuseibeh:2000} phase emphasizes checking for
conflicts and consistency of the requirements. It also makes sure that the
requirements are clear and complete. Additionally, the agreed upon requirements
are documented in the documentation and verification phase. This documentation
has a clear and precise definition of the system functionalities that acts as an
agreement between stakeholders and developers. These requirements are
documented, usually as natural language, diagrams or mathematically formulae.
Such documents are used and iterated upon until the end of the project.

System requirements are classified as business requirements, user
requirements, functional requirements (\FR) and non-functional requirements
(\NFR). \FR are the system requirements that include the
main features and characteristics of the desired system. \NFR are the
system's properties and constraints \cite{Davis:1993, Glinz}. \NFR set the
criteria for judging the operation of the system e.g.
performance, availability or reliability. Business requirements are specified to
address a business' objectives, vision, and goals. They are defined at a high
level to keep \levi{preserve? Don't get this sentence} the knowledge from organization
or company side for designing the products.
User requirements are the users' wish list for the system. User requirements are
valuable for ensuring that system performs as the users wished it to.
