\subsection{Requirements Engineering}

Software systems are developed over millions of lines of code, number of modules and
documents. The primary goal of the software system is to satisfy users by developing the
software that can meet their needs and expectations. This goal is achievable by applying
different methodologies and engineering techniques. One of the key factor is to understand and
identify the needs of users, also known as, software requirements. Software requirement
engineering is the process that helps to determine the requirements in a systematic way to know
what functionalities the targeted system should have to fulfil user's needs. Formally RE is
defined as \cite{Zave:1997:CRE:267580.267581}:\\
\begin{quote}
``Requirements engineering is the branch of software engineering concerned with the real-world
goals for, functions of, and constraints on software systems. It is also concerned with the
relationship of these factors to precise specifications of software behavior, and to their evolution
over time and across software families."
\end{quote}
Software requirements play a key role in the success of a project. In the USA, a survey was
conducted over 8380 projects by 350 companies to know the project failure rates. The report
overall results showed only 16.2\% projects were completed successfully and one-half (52.7\%)
of projects met with challenges and were completed with partial functionalities, time delays and
over budget. Almost 31\% of the projects were never completed. The main cause told by the
executive managers was the poor requirement. The major problems were the lack of user
involvement (13\%), requirements incompleteness (12\%), changing requirements (11\%),
unrealistic expectations (6\%), and unclear objectives (5\%).\cite{knuthwebsite} [6]\\
Software requirement engineering has mainly four phases; requirement elicitation, requirement
analysis, requirement documentation and requirement verification \cite {Kotonya:1998:REP:552009}. Requirement elicitation
\cite{Coughlan:2002:ECR:2740304.2740444,Zowghi2005} helps to understand the stakeholders needs, e.g. what features he wants in the software.
Requirement elicitation techniques are mostly derived by the social sciences, organizational
theory, knowledge engineering and practical experience. For requirements elicitation, different
techniques exist in the literature that include interviews, questioners and ethnography etc.
Requirement analysis \cite{Nuseibeh:2000:RER:336512.336523} is the next step after requirement elicitation. In this phase, software
requirements are analyzed to check conflicts and consistency of requirements. It is also makes
sure that the requirements are clear, complete and consistent. Furthermore, the agreed
requirements are documented. This documentation has a clear and precise definition of the
system functionalities. It also acts as an agreement between stakeholders and developers. These
functionalities and requirements are documented usually as diagrams, mathematically formulae
or natural languages. These documents are used and iterated until the end of the projects.
System requirements are classified into businees requirments, user requirments, functional requirements (FR) 
and non-functional requirements (NFR). Functional requirements are the system requirements that include the main
features and characteristics of the desired system. Non-functional requirements are the system
properties and constraints \cite{Davis:1993:SRO:113586}. NFRs set the criteria for judging the operation of the system e.g.
performance, availability and reliability etc.