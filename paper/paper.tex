\documentclass[conference]{IEEEtran}
\IEEEoverridecommandlockouts

\usepackage{cite}
\usepackage{amsmath,amssymb,amsfonts}
\usepackage{algorithmic}
\usepackage{graphicx}
\usepackage{textcomp}
\usepackage{xcolor}
\def\BibTeX{{\rm B\kern-.05em{\sc i\kern-.025em b}\kern-.08em
    T\kern-.1667em\lower.7ex\hbox{E}\kern-.125emX}}

\usepackage{calc, amssymb, amstext, amsmath}
\usepackage{boxedminipage, xspace, here, float}
\usepackage{epsfig, graphics, graphicx, xcolor, color, setspace, tabularx, multirow}
\usepackage[linewidth=1pt]{mdframed}
\usepackage{chngcntr} 
\usepackage[font={small}]{caption, subfig}
\usepackage{balance}

\usepackage{hyperref, url}
\usepackage{arydshln} 
\usepackage{csquotes}

\pdfcompresslevel=9 
%\DeclareGraphicsExtensions{.png}  
\graphicspath{{img/}{svg/}} 

 
 
\usepackage{ifthen} 
\newboolean{showcomments} 
\setboolean{showcomments}{true} % toggle to show or hide comments
\ifthenelse{\boolean{showcomments}} 
  {\newcommand{\nb}[2]{
    \fcolorbox{gray}{yellow}{\bfseries\sffamily\scriptsize#1}  
    {$\blacktriangleright$#2$\blacktriangleleft$} 
   }
   \newcommand{\version}{\emph{\scriptsize$-$working$-$}}
  } 
  {\newcommand{\nb}[2]{} 
   \newcommand{\version}{}  
  }    

% for comments 
\newcommand\levi[1]{\nb{Levi}{\textcolor{teal}{#1}}}
\newcommand\tahira[1]{\nb{Tahira}{\textcolor{blue}{#1}}}
\newcommand\parisa[1]{\nb{Parisa}{\textcolor{red}{#1}}} 
\newcommand\saad[1]{\nb{Saad}{\textcolor{red}{#1}}} 
 
\newcommand{\etal}{\emph{et al.}\xspace}
\newcommand{\RE}{\textsc{Re}\xspace} 
\newcommand{\ML}{\textsc{Ml}\xspace} 
\newcommand{\NLP}{\textsc{Nlp}\xspace}
\newcommand{\DOORS}{\textsc{Doors}\xspace}


%\setlength{\textwidth}{13cm}
%\setlength{\textheight}{19cm}
% \setlength{\oddsidemargin}{1.1cm}  
% \setlength{\evensidemargin}{1.1cm}

%\newcommand{\newpar}[1]{\noindent 
%\textbf{#1.}\hspace{0.3cm}
%}

\usepackage{inputenc} 
\inputencoding{utf8} 

\begin{document}

\title{A Bird’s Eye View on Requirements Engineering and Machine learning*\\
{\footnotesize \textsuperscript{*}Note: Sub-titles are not captured in Xplore and
should not be used}
\thanks{Identify applicable funding agency here. If none, delete this.}
}

\author{\IEEEauthorblockN{1\textsuperscript{st} Tahira Iqbal}
\IEEEauthorblockA{\textit{fortiss GmbH} \\ Munich, Germany \\
iqbal@fortiss.org}
\and
\IEEEauthorblockN{2\textsuperscript{nd} Parisa Elahidoost} 
\IEEEauthorblockA{\textit{fortiss GmbH} \\
Munich, Germany \\
elahidoost@fortiss.org}
\and
\IEEEauthorblockN{3\textsuperscript{rd} Levi Lucio}
\IEEEauthorblockA{\textit{fortiss GmbH} \\
Munich, Germany \\
lucio@fortiss.org}} 
\maketitle

\begin{abstract}
Machine learning (ML) has demonstrated practical impact in a variety of application domains. Software engineering is a fertile domain where ML is helping in automating  different tasks. In this paper, our focus is the intersection of software requirement engineering (RE) and ML.  To obtain an overview of how ML is helping RE and the research trends in this area, we have surveyed a large number of research articles. We found that the impact of ML can be observed in requirement elicitation, analysis and specification, validation and management. Furthermore, in these categories, we discuss the specific problem solved by ML, the features and ML algorithms used, and datasets, when available. We outline lessons learned and envision possible future directions for the domain.
\end{abstract}

\begin{IEEEkeywords}
component, formatting, style, styling, insert
\end{IEEEkeywords}
\section{Introduction}

This is an introduction~\cite{Domingos:2015}

\subsection{Machine Learning}

Machine-Learning (\ML) \cite{mitchell:1997} is a range of algorithm to
approximate functions and discover patterns in data.
Historically, models and heuristics are human-built exhaustive prescriptions of
how a system should behave. \ML is grounded on different premises:
rather than relying on humans to input all the possible cases the system can
handle, the field attempts to extrapolate patterns from a representative
set of examples that illustrate the expected behaviors. The way in which a
learning algorithm operates attempts to emulate the way in which humans learn:
from a set of examples, a general model for a behavior is induced.

Many learning algorithms exist, based on different visions of how learning
happens in practice~\cite{Domingos:2015}. All these algorithms have in common
the notion of \emph{features}. Features correspond to characteristics of what is
being learned and provide the grounds for the algorithm to abstract from the
complexities of the real world. Assume for example that an algorithm should
learn, based on a brain scan of a medical patient, to decide whether that
patient has brain cancer or not. A number of \emph{features} such as for example
the ``number of irregular objects in the scan'', the ``color of such objects'',
the ``disposition of such objects'' would be provided to the algorithm.
Additionally, the algorithm is fed a number of brain scans together with
decisions previously taken on them (cancer found / cancer not found) -- the
\emph{training data}.
The learning algorithm then undergoes a \emph{training phase}. It attempts to
find an internal model that allows it to map the decisions to the brain scans,
given the training data. The model obtained from the training step is useful if
it performs well (generalizes) when applied to new data from outside the training
set -- in our example, when it can accurately diagnose brain cancer for new
brain scans. Such generalization is based on the premise that inputs that are
``closer'', in terms of the given \emph{features}, should lead to ``closer''
outputs.

The formal notion of ``closeness'' is a characteristic of the learner
algorithm being employed and determines how the algorithm generalizes the computation from
the given examples. Achieving good generalizations is the cornerstone of
machine learning and \emph{overfitting} (performing very well on training
inputs but very poorly on new inputs) is one of its major challenges.\levi{cut
these last couple of sentences}

More formally, in textbooks, courses and articles, Machine Learning is often
defined following the definition of Tom Mitchell~\cite{mitchell1997}:
\begin{quote}
	A computer program is said to learn from experience E with respect to some class of tasks T and performance measure P if its performance at tasks in T, as measured by P, improves with experience E.
\end{quote}
Therefore, it is said that to classify some patients into classes (e.g healthy and unhealthy), the task T, one have to define an algorithm that provides a model, such as an artificial neural network. The quality of this model is quantified by a measure P, for instance its accuracy while predicting the classes. This measure is then sent back to the algorithm, a new experience E, in order to choose or improve the model.
A machine learning tasks can be discussed and subdivided based on the elements
of the following equation:
\begin{equation*}
	f(\mathsf{\bf X}) = \mathsf{\bf y} + \xi
\end{equation*}
where $\mathsf{\bf X}$ is the $n \times d$ input matrix, containing $n$ samples
characterized by $d$ features, $\mathsf{\bf y}$ is the $n \times 1$ target
vector containing the classes of the $n$ samples and $\xi$ is a $n \times 1$
vector representing the noise. The goal is to approximate $f$ in order to
provide the best mapping between $\mathsf{\bf X}$ and $\mathsf{\bf y}$, given
some noise $\xi$. Indeed, the approximation has to map a $\mathsf{\bf X}$
containing noise, to a $\mathsf{\bf y}$ which may contain noise too. For
instance, uncontrolled conditions such as the room temperature and the exposure
time to this temperature can induce variations in the information contained in 
collected blood samples. Moreover, the $\xi$ term also contains the
approximation error when, for example, one tries to approximate a non-linear
function with a linear function.

The problem presented by the above equation is called \emph{supervised learning}, and can be roughly subdivided
in two popular problems: \emph{classification} and \emph{regression}.
When the target vector ${\bf y}$ is composed of categorical values (i.e. classes), then we have a
classification problem. The goal is to learn how to link instances or samples in
${\bf X}$ to a certain class (e.g. healthy patient or unhealthy patient).
However, if the target vector contains continuous values, we face a regression
problem (e.g. predict the body temperature of a patient given some clinical features of the
patient).

In some cases, ${\bf y}$ is not given and we have to find patterns in ${\bf X}$
``blindly.'' This is called an \emph{unsupervised} learning problem. Finding
clusters in ${\bf X}$, i.e. finding a ${\bf y}$ that has never been given, is such a
problem. For instance, one may want to group patients based on symptoms they have.

\emph{Reinforcement Learning} can be seen as
an intermediate problem where ${\bf y}$ is not given but the procedure is guided nevertheless. In RL, an agent has
to find a sequence of actions leading to a success. The fact that the sequence
leads to a success, i.e what would be in ${\bf y}$, is not known in advance, but
rewards are given to the agent in order for him to know if it follows a path to
success. In other words, the goal is, by providing rewards along the way, to
find the sequence of actions leading to the desired state. Typical examples can
be found in gaming, where an agent receives a reward when he wins the game. From
the different chains of actions that led him to a reward, the agent must generalize to find how to win that game.
\subsection{Requirements Engineering}

Software systems are developed over millions of lines of code, number of modules
and documents. The primary goal of the software system is to satisfy users by
developing the software that can meet their needs and expectations. This goal is
achievable by applying different methodologies and engineering techniques. One
of the key factor is to understand and identify the needs of users, also known
as, software requirements. Software requirement engineering is the process that
helps to determine the requirements in a systematic way to know what
functionalities the targeted system should have to fulfil user's needs. Formally
RE is defined as \cite{Zave:1997:CRE:267580.267581}:\\
\begin{quote}
``Requirements engineering is the branch of software engineering concerned with the real-world
goals for, functions of, and constraints on software systems. It is also concerned with the
relationship of these factors to precise specifications of software behavior, and to their evolution
over time and across software families."
\end{quote}
Software requirements play a key role in the success of a project. In the USA, a
survey was conducted over 8380 projects by 350 companies to know the project
failure rates. The report overall results showed only 16.2\% projects were
completed successfully and one-half (52.7\%) of projects met with challenges and
were completed with partial functionalities, time delays and over budget. Almost
31\% of the projects were never completed. The main cause told by the executive
managers was the poor requirement. The major problems were the lack of user
involvement (13\%), requirements incompleteness (12\%), changing requirements
(11\%), unrealistic expectations (6\%), and unclear objectives
(5\%).\cite{knuthwebsite} [6]\\
Software requirement engineering has mainly four phases; requirement
elicitation, requirement analysis, requirement documentation and requirement
verification \cite {Kotonya:1998:REP:552009}. Requirement elicitation
\cite{Coughlan:2002:ECR:2740304.2740444,Zowghi2005} helps to understand the
stakeholders needs, e.g. what features he wants in the software.
Requirement elicitation techniques are mostly derived by the social sciences,
organizational theory, knowledge engineering and practical experience. For
requirements elicitation, different techniques exist in the literature that
include interviews, questioners and ethnography etc.
Requirement analysis \cite{Nuseibeh:2000:RER:336512.336523} is the next step
after requirement elicitation. In this phase, software requirements are analyzed
to check conflicts and consistency of requirements. It is also makes sure that
the requirements are clear, complete and consistent. Furthermore, the agreed
requirements are documented. This documentation has a clear and precise
definition of the system functionalities. It also acts as an agreement between
stakeholders and developers. These functionalities and requirements are
documented usually as diagrams, mathematically formulae or natural languages.
These documents are used and iterated until the end of the projects.
System requirements are classified into businees requirments, user requirments,
functional requirements (FR) and non-functional requirements (NFR). Functional
requirements are the system requirements that include the main features and
characteristics of the desired system. Non-functional requirements are the
system properties and constraints \cite{Davis:1993:SRO:113586}. NFRs set the
criteria for judging the operation of the system e.g.
performance, availability and reliability etc.\levi{what about the business and
user requirements. Do we consider them?}
  
\section{Contributions}

\levi{summarize preprocessing steps and feature calculation}
\levi{disclaimer: sometimes features are presented sometimes not} 
\section{Discussion}

\begin{itemize}
  \item Not better than SOA, comparations with humans, comparison with non-ML
  techniques
  \item There are a number of datasets available
  \item Clustering is on par with classification
\end{itemize}

\newcommand\nocell[1]{\multicolumn{#1}{c|}{}}
\begin{table*}%
	\begin{scriptsize}
\begin{center}
	\begin{tabular}{|c|l|l|l|l|l|}
		\cline{2-6}
		\nocell{1} & \textbf{Themes} & \textbf{Contributions} & \textbf{ML Task} &
		\textbf{\ML Model Types} & \textbf{Datasets
		Used}\\
		\cline{2-6}
		\cline{1-6} 
		\multirow{3}{*}{\rotatebox[origin=c]{90}{\textbf{E}}}
			& External  & \cite{Guzman:2017}($\circ$)  \cite{Williams:2017}(+)
			\cite{Jiang:2014}(+) \cite{Douglas:S2008}($\circ$) \cite{Jha:2017}(+) 
			\cite{Castro-Herrera:2009}($\circ$) & Classification / Clustering &
			(Multinomial) Na\"ive Bayes; & Online reviews for KIS 2011 (from Amazon) and TuneIn 3.6 (from App Store)
			\\&&&& Support Vector Machines & Skiweb data\\
			\cdashline{2-6}[.4pt/1pt]
		\hline
		\multirow{3}{*}{\rotatebox[origin=c]{90}{\textbf{A}}} 
			& Non-Functional & ~\cite{Slankas:2013}($\circ$) ~\cite{Cleland-Huang2007}($\circ$)& Classification & k-Nearest Neighbors & Open Source PROMISE Datase\footnote{http://promise.site.uottawa.ca/SERepository/datasets-page.html}\\ &&&&& Siemens Logistics and Automotive Organization requirement documents \\
			\cdashline{2-6}[.4pt/1pt]
			& Functional & ~\cite{7949577}($\circ$) & Classification & Bi-Directional Long Short-Term Memory\\&&&&Conditional Random Field Network & E-commerce Software Specification Documents \\
			\cdashline{2-6}[.4pt/1pt]
			& Functional \& Non-Functional & \cite{Lu:2017}($\circ$) \cite{Deocadez:2017}($\circ$) \cite {Kurtanovic:2017}($\circ$) \cite{Abad:2017}($\circ$) \cite {Garzoli:2013}($\circ$) ~\cite{Wieloch:2013}($\circ$)  & Classification  &  Bagging, Naive Bayes, SVM & Open Source PROMISE Datase, appstore reviews, internal dataset \\
			\cdashline{2-6}[.4pt/1pt]
			& Prioritization & \cite{S.Dhingra}($\circ$) \cite{PAvesani}(+) \cite{7320432}($\circ$) & Classification & Case Based Ranking \ J.48 DT & Software Requirements Documents in Natural Language\\
			\cdashline{2-6}[.4pt/1pt] 
			& Security & & & & \\
			\cdashline{2-6}[.4pt/1pt]
			& Contextuals & & & & \\
		\hline
		\multirow{3}{*}{\rotatebox[origin=c]{90}{\textbf{V}}} 
			& Traceability & ~\cite{Gervasi:2011}($\circ$) ~\cite{Sultanov:2013}($\circ$) & Classification / Reinforcement Learning & Naive Bayesian / J48 Decision-Tree & Open Source CM-1 NASA project\footnote{http://promise.site.uottawa.ca/SERepository/datasets/cm1.arff}\\ &&&&& Open Source Pine Dataset\footnote{http://www.washington.edu/pine/} \\
			\cdashline{2-6}[.4pt/1pt]
		\hline
		\multirow{3}{*}{\rotatebox[origin=c]{90}{\textbf{M}}}
			& Visualization & ~\cite{Reddivari:2012}($\circ$) ~\cite{Pinqui:2015}($\circ$) ~\cite{Lucassen:2016}($\circ$) & Classification / Clustering & Support Vector Machines / Ward's method& - \\
			\cdashline{2-6}[.4pt/1pt]
			& Structuring & ~\cite{Duan:2007}($\circ$) ~\cite{Winkler:2016}($\circ$) ~\cite{Rauf:2011}($\circ$) \\ && ~\cite{Ferrari:2013}($\circ$) ~\cite{Ott:2013}($\circ$) & Classification / Clustering & Multinomial Naive Bayes / Support Vector Machines \\ &&&& Convolutional Neural Networks \\ &&&& Sliding Head-Tail
Component Clustering \\ &&&& Hierarchical Clustering & International Union of
Railways: EIRENE Functional Requirements Specification\footnote{https://www.transportstyrelsen.se/globalassets/global/jarnvag/vagledning/godkannande/eirene($\_$)frsv7.pdf}\\ &&&&& Mercedes-Benz passenger car development Specifications\footnote{http://wwwbroy.in.tum.de/lehre/vorlesungen/ase/ss05/iese-002($\_$)02.pdf} \\
			\cdashline{2-6}[.4pt/1pt]
		\hline
	\end{tabular}
\end{center}
(+) improves the state of the art; (-) comparable to or worse
	than state of the art; ($\circ$) no information on how the approach
	relates to the state of the art\\\textbf{E}: Elicitation, \textbf{A}: Analysis, \textbf{V}: Verification
	\textbf{M}: Management
	\end{scriptsize}
	\caption{Contributions and \ML tasks related to each theme within each \RE
	approach.}
	\label{tab:summary}
\end{table*}


\levi{internal vs external} 
\section{Threats to Validity}
\label{sec:threats_to_validity}

The validity of the study might be affected by the coverage of the search results, bias on study selection, and inaccuracy of data extraction.
\paragraph{Study Coverage.}
Some relevant studies could be missing in our study due to inadequate search strings or missing databases.
To cope with this threat, the data preparation was based on a systematic method.
\vspace{-5pt}
\paragraph{Study Selection Bias.}
Study assessment might be biased by researchers.
To mitigate for this threat, a set of include and exclude criteria was predefined and researchers assessed the title and abstract of the papers based on them to steer the assessment.
\vspace{-5pt}
\paragraph{Inaccuracy of Data Extraction.}
Also the data extraction process might be biased by researchers.
To mitigate for this threat, the selection of data items was strictly driven by the research questions.
Moreover, assignments were marked by the researchers depending on their confidence level. Low-confidence assignments were discussed between the authors until a consensus was reached.   
\section{Conclusion}
\label{sec:conclusion}

Through our bird's eye view of \ML applied to \RE we have observed that in the
past couple of decades a good amount of research has been done on how to bring
these two worlds together. The stakes are high: while requirements engineering
is currently a domain under intensive research, attempts to address its
challenges academically have translated into few results in practice. Free-form
text-based tools with light-weight structuring capabilities such as DOORS are
now the norm in practice. Requirements elicitation, analysis, validation and
management keep on relying on human expertise and talent. While academics often insist that
better formalization brings advantages, the languages in which requirements are
formalized do not match the need that stakeholders in the \RE process
(technical and non-technical) need to communicate through artifacts that are
intelligible to all.

While not overstating the potential of \ML, which is has its own challenges to
overcome such as coarseness of the learned models, overfitting or hungriness
for data, we have provided in this article indications that \ML might become a cornerstone in \RE. For now, it
seems like the domain is now undergoing a pre-scientific phase: the studies we
have analyzed seldom compare themselves with the state-of-the-art (see
\tab{summary}). This suggests that the current body of research is composed of
new ideas, which have not yet been validated to its full extent by the
scientific or industrial communities. We thus call for a more extensive survey to validate
the preliminary conclusions we present in this work.
 
  
\bibliographystyle{abbrv} 
\bibliography{./ml_req}  

\end{document}
