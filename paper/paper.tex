\documentclass[conference]{IEEEtran}
\IEEEoverridecommandlockouts

\usepackage{cite}
\usepackage{amsmath,amssymb,amsfonts}
\usepackage{algorithmic}
\usepackage{graphicx}
\usepackage{textcomp}
\usepackage{xcolor}
\def\BibTeX{{\rm B\kern-.05em{\sc i\kern-.025em b}\kern-.08em
    T\kern-.1667em\lower.7ex\hbox{E}\kern-.125emX}}

\usepackage{calc, amssymb, amstext, amsmath}
\usepackage{boxedminipage, xspace, here, float}
\usepackage{epsfig, graphics, graphicx, xcolor, color, setspace, tabularx, multirow}
\usepackage[linewidth=1pt]{mdframed}
\usepackage{chngcntr} 
\usepackage[font={small}]{caption, subfig}
\usepackage{balance}

\usepackage{hyperref, url}
\usepackage{arydshln}
\usepackage{csquotes}

\pdfcompresslevel=9 
%\DeclareGraphicsExtensions{.png}  
\graphicspath{{img/}{svg/}} 

 
 
\usepackage{ifthen} 
\newboolean{showcomments} 
\setboolean{showcomments}{true} % toggle to show or hide comments
\ifthenelse{\boolean{showcomments}} 
  {\newcommand{\nb}[2]{
    \fcolorbox{gray}{yellow}{\bfseries\sffamily\scriptsize#1}  
    {$\blacktriangleright$#2$\blacktriangleleft$} 
   }
   \newcommand{\version}{\emph{\scriptsize$-$working$-$}}
  } 
  {\newcommand{\nb}[2]{} 
   \newcommand{\version}{}  
  }    

% for comments 
\newcommand\levi[1]{\nb{Levi}{\textcolor{teal}{#1}}}
\newcommand\tahira[1]{\nb{Tahira}{\textcolor{blue}{#1}}}
\newcommand\parisa[1]{\nb{Parisa}{\textcolor{red}{#1}}} 
\newcommand\saad[1]{\nb{Saad}{\textcolor{red}{#1}}} 

\newcommand{\etal}{\emph{et al.}\xspace}
\newcommand{\RE}{\textsc{Re}\xspace}
\newcommand{\ML}{\textsc{Ml}\xspace}


%\setlength{\textwidth}{13cm}
%\setlength{\textheight}{19cm}
% \setlength{\oddsidemargin}{1.1cm}  
% \setlength{\evensidemargin}{1.1cm}

%\newcommand{\newpar}[1]{\noindent
%\textbf{#1.}\hspace{0.3cm}
%}

\usepackage{inputenc} 
\inputencoding{utf8} 

\begin{document}

\title{A Bird’s Eye View on Requirements Engineering and Machine learning*\\
{\footnotesize \textsuperscript{*}Note: Sub-titles are not captured in Xplore and
should not be used}
\thanks{Identify applicable funding agency here. If none, delete this.}
}

\author{\IEEEauthorblockN{1\textsuperscript{st} Tahira Iqbal}
\IEEEauthorblockA{\textit{fortiss GmbH} \\ Munich, Germany \\
iqbal@fortiss.org}
\and
\IEEEauthorblockN{2\textsuperscript{nd} Parisa Elahidoost}
\IEEEauthorblockA{\textit{fortiss GmbH} \\
Munich, Germany \\
elahidoost@fortiss.org}
\and
\IEEEauthorblockN{3\textsuperscript{rd} Levi Lucio}
\IEEEauthorblockA{\textit{fortiss GmbH} \\
Munich, Germany \\
lucio@fortiss.org}} 
\maketitle

\begin{abstract}
Machine learning (ML) has demonstrated practical impact in a variety of application domains. Software engineering is a fertile domain where ML is helping in automating  different tasks. In this paper, our focus is the intersection of software requirement engineering (RE) and ML.  To obtain an overview of how ML is helping RE and the research trends in this area, we have surveyed a large number of research articles. We found that the impact of ML can be observed in requirement elicitation, analysis and specification, validation and management. Furthermore, in these categories, we discuss the specific problem solved by ML, the features and ML algorithms used, and datasets, when available. We outline lessons learned and envision possible future directions for the domain.
\end{abstract}

\begin{IEEEkeywords}
component, formatting, style, styling, insert
\end{IEEEkeywords}
\section{Introduction}

% This is an introduction~\cite{Domingos:2015}
% \noindent
% \textbf{RQ1:} How is \ML used inside \RE tools?
% 
% \noindent
% \textbf{RQ2:} Is using \ML inside \RE tools beneficial?
% 
% \noindent
% \textbf{RQ3}: What \ML task(s) is (are) used for which purpose in \RE?
% 
% \noindent
% \textbf{RQ4}: Which model types are used to perform the \RE tasks?
% 
% \noindent
% \textbf{RQ5}: How are \ML features extracted/selected to guide \RE tasks?
% \parisa{just kept the old RQ for comparision}
\footnote{This work was supported in part by the European Union’s Horizon 2020
Research and Innovation program under grant agreement no. 674875}
Machine learning algorithms have been shown to have considerable practical
importance in many application domains. This is especially true of domains where
large databases are available and a need for exploring some kind of consistency
exists or, domains where a program needs to adapt itself to
changes~\cite{Zhang2002}. Requirements engineering is a critical part of
software engineering and it seems appropriate to use machine learning methods
for requirements engineering tasks. Because requirements specification documents
are mainly given in natural language, \ML can be useful by emulating human
processing.
\\
This paper aims to present a survey of how ML benefits existing RE
approaches. More precisely, we pursue the following research questions: \\
\noindent
\textbf{RQ1:} What is the current state of the practice in \ML${\&}$\RE?

\noindent
\textbf{RQ2:} What types of learning methods are used when \ML is applied to
\RE?

\noindent
\textbf{RQ3:} Which are the \RE problems that currently use \ML methods?

\noindent
\textbf{RQ4:} Is using \ML methods improving \RE?

To reply to these research questions we have performed a literature review,
split into data preparation, data collection, and data analysis phase. First, a search string
was prepared based on the research questions, then a search was performed over a
predefined set of databases and all identified studies were assessed by means of
title and abstract. Our literature review is not meant to be an exhaustive study
of the field -- rather we are offering a snapshot of the current
state-of-the-art by borrowing some techniques from Systematic Literature
Reviewing.\\
The major contributions of this article are as follows:

\begin{itemize}
    \item We provide an overview of the current state of the art of some of the
    challenges \RE faces that may be handled through \ML techniques. We focus on
    two important aspects:
    \begin{itemize}
        \item Providing an overview of the \ML problem categories
        (classification, regression, clustering, etc.) in use for the support of
        \RE tasks (elicitation, analysis, validation and management).
        \item Providing an overview of the common \ML models (decision tree,
        K-Nearest Neighbors, Naive Bayesian, etc.) for tackling \RE problems and
        the data sets if available.
    \end{itemize}
    \item We analyze the literature to discover trends and lessons on the use of \ML in \RE.
\end{itemize}
The paper is organized as follows. The rest of this section provides
background information on \ML and \RE required to understand the remaining of
the paper. Section \ref{sec:contributions} provides an overview of \RE tasks
where \ML has been used. In section \ref{sec:discussion} we summarize the major
findings of our study. Finally, in section \ref{sec:threats_to_validity} we
state the threats to the validity of our work and section\ref{sec:conclusion}
concludes the paper.
\section{Machine Learning}

Machine-Learning (\ML) \cite{mitchell:1997} is a range of algorithm to
approximate functions and discover patterns in data.
Historically, models and heuristics are human-built exhaustive prescriptions of
how a system should behave. \ML is grounded on different premises:
rather than relying on humans to input all the possible cases the system can
handle, the field attempts to extrapolate patterns from a representative
set of examples that illustrate the expected behaviors. The way in which a
learning algorithm operates attempts to emulate the way in which humans learn:
from a set of examples, a general model for a behavior is induced.

Many learning algorithms exist, based on different visions of how learning
happens in practice~\cite{Domingos:2015}. All these algorithms have in common
the notion of \emph{features}. Features correspond to characteristics of what is
being learned and provide the grounds for the algorithm to abstract from the
complexities of the real world. Assume for example that an algorithm should
learn, based on a brain scan of a medical patient, to decide whether that
patient has brain cancer or not. A number of \emph{features} such as for example
the ``number of irregular objects in the scan'', the ``color of such objects'',
the ``disposition of such objects'' would be provided to the algorithm.
Additionally, the algorithm is fed a number of brain scans together with
decisions previously taken on them (cancer found / cancer not found) -- the
\emph{training data}.
The learning algorithm then undergoes a \emph{training phase}. It attempts to
find an internal model that allows it to map the decisions to the brain scans,
given the training data. The model obtained from the training step is useful if
it performs well (generalizes) when applied to new data from outside the training
set -- in our example, when it can accurately diagnose brain cancer for new
brain scans. Such generalization is based on the premise that inputs that are
``closer'', in terms of the given \emph{features}, should lead to ``closer''
outputs.

The formal notion of ``closeness'' is a characteristic of the learner
algorithm being employed and determines how the algorithm generalizes the computation from
the given examples. Achieving good generalizations is the cornerstone of
machine learning and \emph{overfitting} (performing very well on training
inputs but very poorly on new inputs) is one of its major challenges.\levi{cut
these last couple of sentences}

More formally, in textbooks, courses and articles, Machine Learning is often
defined following the definition of Tom Mitchell~\cite{mitchell1997}:
\begin{quote}
	A computer program is said to learn from experience E with respect to some class of tasks T and performance measure P if its performance at tasks in T, as measured by P, improves with experience E.
\end{quote}
Therefore, it is said that to classify some patients into classes (e.g healthy and unhealthy), the task T, one have to define an algorithm that provides a model, such as an artificial neural network. The quality of this model is quantified by a measure P, for instance its accuracy while predicting the classes. This measure is then sent back to the algorithm, a new experience E, in order to choose or improve the model.
A machine learning tasks can be discussed and subdivided based on the elements
of the following equation:
\begin{equation*}
	f(\mathsf{\bf X}) = \mathsf{\bf y} + \xi
\end{equation*}
where $\mathsf{\bf X}$ is the $n \times d$ input matrix, containing $n$ samples
characterized by $d$ features, $\mathsf{\bf y}$ is the $n \times 1$ target
vector containing the classes of the $n$ samples and $\xi$ is a $n \times 1$
vector representing the noise. The goal is to approximate $f$ in order to
provide the best mapping between $\mathsf{\bf X}$ and $\mathsf{\bf y}$, given
some noise $\xi$. Indeed, the approximation has to map a $\mathsf{\bf X}$
containing noise, to a $\mathsf{\bf y}$ which may contain noise too. For
instance, uncontrolled conditions such as the room temperature and the exposure
time to this temperature can induce variations in the information contained in 
collected blood samples. Moreover, the $\xi$ term also contains the
approximation error when, for example, one tries to approximate a non-linear
function with a linear function.

The problem presented by the above equation is called \emph{supervised learning}, and can be roughly subdivided
in two popular problems: \emph{classification} and \emph{regression}.
When the target vector ${\bf y}$ is composed of categorical values (i.e. classes), then we have a
classification problem. The goal is to learn how to link instances or samples in
${\bf X}$ to a certain class (e.g. healthy patient or unhealthy patient).
However, if the target vector contains continuous values, we face a regression
problem (e.g. predict the body temperature of a patient given some clinical features of the
patient).

In some cases, ${\bf y}$ is not given and we have to find patterns in ${\bf X}$
``blindly.'' This is called an \emph{unsupervised} learning problem. Finding
clusters in ${\bf X}$, i.e. finding a ${\bf y}$ that has never been given, is such a
problem. For instance, one may want to group patients based on symptoms they have.

\emph{Reinforcement Learning} can be seen as
an intermediate problem where ${\bf y}$ is not given but the procedure is guided nevertheless. In RL, an agent has
to find a sequence of actions leading to a success. The fact that the sequence
leads to a success, i.e what would be in ${\bf y}$, is not known in advance, but
rewards are given to the agent in order for him to know if it follows a path to
success. In other words, the goal is, by providing rewards along the way, to
find the sequence of actions leading to the desired state. Typical examples can
be found in gaming, where an agent receives a reward when he wins the game. From
the different chains of actions that led him to a reward, the agent must generalize to find how to win that game.
\subsection{Requirements Engineering}

Software systems are developed over millions of lines of code, number of modules
and documents. The primary goal of the software system is to satisfy users by
developing the software that can meet their needs and expectations. This goal is
achievable by applying different methodologies and engineering techniques. One
of the key factor is to understand and identify the needs of users, also known
as, software requirements. Software requirement engineering is the process that
helps to determine the requirements in a systematic way to know what
functionalities the targeted system should have to fulfil user's needs. Formally
RE is defined as \cite{Zave:1997:CRE:267580.267581}:
\begin{quote}
``Requirements engineering is the branch of software engineering concerned with the real-world
goals for, functions of, and constraints on software systems. It is also concerned with the
relationship of these factors to precise specifications of software behavior, and to their evolution
over time and across software families."
\end{quote}

	Software requirements plays a key role in the success of a project. In the USA, a
survey was conducted over 8380 projects by 350 companies to know the project
failure rates. The report \cite{StandishReport} results showed only 16.2\% projects were
completed successfully and one-half (52.7\%) of projects met with challenges and
were completed with partial functionalities, time delays and over budget. Almost
31\% of the projects were never completed. The main cause identified by
executive managers was poor requirements engineering. The main culprits were the lack of
user involvement (13\%), requirements incompleteness (12\%), changing requirements
(11\%), unrealistic expectations (6\%), and unclear objectives
(5\%). 

	Software requirement engineering has mainly four phases; \emph{requirement
elicitation}, \emph{requirement analysis}, \emph{requirement documentation} and
\emph{requirement verification} \cite {Kotonya:1998}. Requirement elicitation
\cite{Coughlan:2002, Zowghi2005} helps to understand the
stakeholders needs, e.g. what features he wants in the software.
Requirement elicitation techniques are mostly derived by the social sciences,
organizational theory, knowledge engineering and practical experience. For
requirements elicitation, different techniques exist in the literature that
include interviews, questionnaires and ethnography etc. to understand the
stakeholder requirements.
Requirement analysis \cite{Nuseibeh:2000} is the next step
after requirement elicitation. In this phase, software requirements are analyzed
to check for conflicts and consistency of requirements. It also makes sure
that the requirements are clear and complete.
Furthermore, the agreed requirements are documented. This documentation has a clear and precise
definition of the system functionalities. It also acts as an agreement between
stakeholders and developers. These requirements are documented usually as diagrams, mathematically
formulae or natural languages.These documents are used and iterated upon until the end of the projects.

	System requirements are classified into business requirements, user
requirements, functional requirements (FR) and non-functional requirements (NFR). Functional
requirements are the system requirements that include the main features and
characteristics of the desired system. Non-functional requirements are the
system properties and constraints \cite{Davis:1993, Glinz}. NFRs set the
criteria for judging the operation of the system e.g.
performance, availability or reliability. Business requirements are specified to
address business objectives, vision, and goals. It is defined at a high level to
keep the knowledge from organization or company side for designing the products.
User requirements are the wish list for the system from users\levi{I don't
think the intersection between FRs, NFRs, BRs and URs is empty}.
User requirements are valuable for ensuring that system performs similarly as users wanted it to do.

\subsection{Text Prepration for ML}
\label{sec:preprocessing}
\levi{summarize preprocessing steps and feature calculation}
\levi{disclaimer: sometimes features are presented sometimes not}
\levi{ Stakeholders needs
(features)\levi{features in this case are just words} are initially preprocessed
by removing common (stop) terms, and computation of tf-id ,explain tf–idf,
maybe extract it to a common section, in general explain that features are
words and the instances are tweets/reviews/\ldots for each term.}

Requirements are written in the natural language such as documents or reviewing
some application. These can be in a variety of forms from a list of individual
words, sentences, multiple paragraphs, short texts with special characters or
others. Before applying a machine learning algorithm on them different steps
employed to transform words into features such as text mining, and natural langugae processing (NLP). NLP text preprocessing phase relies
majorly on pre-built dictionaries, databases, and rules. The common
preprocessing steps in our literature include capitalization, tokenization,
lemmatization, stop words removal, stemming, part of speech (POS). Tokenization
is the process of splitting paragraphs into sentences, or sentences into words.
Capitalization brings everything to lower case for simplicity. Stop words
removal removes all connecting words such as ``and'', ``the'' or others by
comparing the text to a list of stopwords. POS takes text and assigns each part of speech to each
word that helps to build more understanding of a text. Stemming is a process
where words are reduced to a root by removing the unnecessary suffix e.g. eating
after stemming is eat. Lemmatization is an alternative approach of stemming which
able to capture canonical forms based on a word's lemma. It uses part of speech
and WordNet’s lexical database of English for removing inflection. The word
better, stemming will fail to provide any lemma but lemmatization would result
good.

 	Another way to extract features from the text is Bag of Words (BoW). It is a
 model used in natural language processing and categorizes documents based on a
 vocabulary of words and occurrence of words. The common used NLP method is
 Vector Space Modeling (VSM). It is a way to represent documents into
 multidimensional space for information retrieval and documents classification
 and clustering. For example, if you give a query then VSM will find relevant items
 or term from a corpus.

\tahira{ Hollis \etal \cite{Hollis2017} proposed an initial study to automate requirement
elicitation in Agile environment by providing the list of words and loosely
formatted list of requirements. The proposed methodology applied text mining
technique on recorded conversation of the stakeholder and developer
conversation. It was a short paper and did not provide details.Dong \etal \cite{dong2010} also applied text mining on different form of
document and resources from internet for gathering requirement. The system
applied data preprocessing as word segmentation and stop words removal and build
up the VSM model. Kaiya \etal \cite{Kaiya:2010} proposed a tool to improve the
domain knowledge ontology for requirement elicitation by using web mining and
NLP technique. It helped to mine the general concepts to ontology for
requirements elicitation.}

  
\section{Contributions}

\levi{summarize preprocessing steps and feature calculation}
\levi{disclaimer: sometimes features are presented sometimes not}
\subsection{Requirements Elicitation and Discovery}

The manual process of requirement elicitation is expensive in terms of
effort and resources. A project's success majorly depends
on the precise identification of stakeholder's expectations and requirements for
the system they desired.
A possibility to do requirements elicitation is to mine available datasets e.g.
social media, requirement documents, and app stores reviews etc \levi{which documents?}. The latest trend for
identifying user requirements is to mine data obtained from platforms like Twitter,
Google Play Store, and Apple Store etc. These user reviews are not structured requirements
and contain information such as praises, dislikes or bug reports. Also, it contains extra information and noise that make manual requirement elicitation a difficult and challenging task.
Automated requirement elicitation is helpful in these cases and can
significantly reduce time, effort, and cost. This is mainly an ML
\emph{classification} task: give the set of information and identify it as a
requirement or not. Sometimes \emph{clustering} is also used for auxiliary
tasks.\\

Guzman \etal \cite{Guzman:2017} proposed the ALERTme approach for
classifying, grouping and ranking tweets during the software
evolution development. Many users shared their opinions about various software on Twitter. The huge amount of dataset made it hard to manually identify tweets that contained user requirements. The proposed methodology classified tweets as improvement requests or not, using \emph{Naive Bayes} algorithm. This was the first study of
its kind that was performed on software related tweets. 
The classifier was trained with following steps: 1) conversion of
pre-processed tweets into a VSM model, 2) train a classifier
on a set of manually annotated tweets, 3) predict the tweets categories using trained classifier. Furthermore, improvement
requests were considered for the grouping which helped to sort the
requests and summarize them accordingly. The summarization process contained highly ranked tweets based on parameters including likes, sentiments, and number of shares etc.\\

Williams \etal \cite{Williams:2017} performed a similar study on tweets in order
to classify them as user requirements. It used basic pre-processing techniques and applied VSM on data. For the learning
process, manually annotated (labelled) tweets were used and \emph{Naive Bayes} algorithm was applied for classification.\levi{annotated how, which ML
algorithm was used?}. The authors claimed with the help of results that
software tweets are neutral in nature, meaning sentiment analysis did not
influence the outcome of the ML algorithm. It showed improved results in comparison to \cite{Guzman:2017}.
The study used 4000 randomly selected tweets from ten different
softwares including Microsoft Visual Studio, Google Chrome, and Instagram etc\levi{software what?}.\\

Jiang \etal \cite{Jiang:2014} mined user reviews from 
app stores for discovering evolutionary requirements. It first extracted opinions about software features from reviews. For automated opinion identification, syntactic
relation based propagation approach was used that extracted targets and sentiment words
iteratively using known and extracted words. Afterwards, it applied k-mean clustering for opinion categorization.
\levi{explain what kind of clustering}. The proposed system also helped developers to decide requirements related to software revenue by considering economic factors. \levi{explain} It used two datasets of online reviews: one from the
Karplersky internet security 2011 software package (from Amazon) with 380
reviews; the other one  comprising 461 reviews for the  TuneIn Radio Pro V3.6
mobile app (from the app store).\\

Lange \etal\cite{Douglas:S2008} mapped the software requirement elicitation
process onto an existing military tool \emph{skiweb}. Skiweb was used to make decisions about what actions need to be taken on a military command.\levi{which decision?}. Different users posted and updated events and information using this tool. The goal of adding learning capability was to find additional information relevant to user posts. The proposed recommender system used supervised
\emph{Naive Bayes} algorithm to classify text documents in order to find related
requirements to the post \levi{related to what?}. Furthermore, it
\levi{which recommender system?} used topic modeling to identify the key
stakeholders and suggested them requirements for further analysis according to their interest\levi{which analysis?}. This study used an internal organizational
dataset Skiweb Data such as wiki, blogs etc.\\

Jha \etal \cite{Jha:2017} discovered user requirements by mining app store reviews. The requests were classified into three categories;
\emph{bugs}, \emph{features}, and \emph{junk}. The proposed methodology applied Naive Bayes and SVM. The distinction between types of sentences was identified by frame semantics \emph{explain frame semantics} instead of text classification methods. It generated frames for each review, rather than each word. Due to small number of features, a slower dimensional model was produced with enhanced prediction capabilities. It combined existing
datasets from past studies and reviews for iOS apps including CreditKarma,
Fitbit, and Gmail.

Maalej presented in \cite{Maalej} a study on how to classify app reviews as bug
reports, feature requests, user experiences, and ratings. It used
\emph{Naive Bayes} algorithm due to better results in comparison to other algorithms for classification. It also highlighted that binary
classifier performed better than multi
classifiers. It used meta model to enhance the classification performance e.g. ratings,
tense, and sentiment scores etc. A
dataset of 4400 manually annotated reviews from Google Play Store and the Apple App
Store were used for the study.\\

Herrera \etal \cite{Castro-Herrera:2009} built a semi-automated\levi{why
semi-automated?} recommender system to manage a large number of stakeholders participation in
the requirements elicitation and prioritization process. In this system, stakeholders could work collaboratively to transform their needs into sets
of articulated and prioritized requirements. It automatically
generated specialized topics for building forums for stakeholders collaboration and discussion. Stakeholders interests were depicted from their user profiles, that also helped to create recommendations according to the
interest of a community of similar stakeholders.\levi{is this the recommender
part?} For identifying topics, an unsupervised agglomerative clustering algorithm was
applied to unstructured data\levi{what's the difference?}. The proposed system analyzed online
datasets\levi{which datasets?} that were gathered from stakeholders in natural language. 
The evaluation dataset was a collection of 36 feature requests created by
graduate-level students for an Amazon-like student web-portal system.







\subsection{Requirements Specification and Analysis}

\subsubsection{Non-Functional}

Non-functional requirements may not be explicitly mentioned in a formal specification requirements documents even though, all systems have them~\cite{Slankas:2013}. Moreover, freeform documents like interview notes, meeting minutes and scattered requirements specifications include non-functional requirements which need to be detected and classified. In order to support analyst in the error-prone task of manually discovering and classifying NFRs machine learning can be useful. Automatic detection can be used to quickly and more effectively analyze large and complex documents for searching the NFRs\cite{Cleland-Huang2007}. This is a classification problem as from a set of requirements we want to decide a class membership. \\

One of the studies is by Slankas \etal~\cite{Slankas:2013} where they automatically identified and classified sentences in natural language from use agreements, install manuals, regulations, request for proposals, requirements specifications, and user manuals output into 14 different NFRs categories: Access Control (AC), Audit (AU), Availability (AV), Legal (LG), Look and Feel (LF), Maintenance (MT), Operational (OP), Privacy (PR), Recoverability (RC), Performance and Scalability (PS), Reliability (RL), Security (SC), Usability (US). Their two-step process: 1) parse natural language and turn sentences into graphs 2) classify sentences into categories led them into finding 20 keywords for each category of NFRs. \newline
Cleland-Huang \etal~\cite{Cleland-Huang2007} provided the same approach and used supervised classification for grouping non-functional requirements: availability, look-and-feel, legal, maintainability, operational, performance, scalability, security, and usability.

\subsubsection{Functional}

Software requirements specifications are usually stated in informal, imprecise and ambiguous natural language, thus analyzing them is a challenging task. However, for requirements reuse in Software Product Line analyzing is a vital task. Automatically extract structured information of functional requirements from Software Requirements Specifications and grouping them into different categories is a machine learning classification task\cite{7949577}. \\

Wang \etal~\cite{7949577} applied a combination of machine learning, natural language processing, and semantic analysis methods for automatically extract non-functional requirements and classify them into 10 different cases: Agentive, Action, Objective, Agent mode, Objmod, Locational, Temporal, Manner, Goal, Constraint. Their framework has four steps: corpus construction,  NLP  preprocessing, feature extraction and  EFRF (Extended Functional Requirements Frame) functional cases extraction. which for NLP processing they did tokenization,   lemmatization, part-of-speech tagging (POS  tagging) and dependency parsing. Ultimately, they showed that their trained model on E-commerce requirements dataset can be used to extract semantic information from the requirements of automaker systems.

\subsubsection{Functional and Non functional Requirements}

The success of system solely not depends on functional requirements. Just as functional requirements, it also significantly depends upon the adherence to non-functional requirements. In general, the primary focus is more for identification and specification of the FR.  NFRs usually identified and specified in the late development process that can increase the risks. FRs tend to be more straightforward e.g. store and retrieve transaction. On the other side, NFRs are complicated and challenging to implement e.g., making the design to meet NFRs or design test case for them. Different types of requirements analyzed in different way and it is useful to have separate division to look at one particular class. That is why it is necessary to distinguish between FR and NFR and categorize NFRs into subcategories. This distinction help to manage changes in requirements. The manual distribution is difficult and time consuming. Machine learning can be used for reducing the effort and categorizing the requirements based on the text segment analysis. This is ML classification task: give the set of requirements and identifying its category. \\

\etal~\cite{} Automatically classify the user review text into FR, NFR and others. It further classify the NFR into four categories as reliability, usability, portability, and performance. It used supervised machine learning algorithm (bagging) for training the classifier. The text was trimmed by stop words elimination, lemmatization, stemming, and sentences split. For augmenting the user review word2vec used. Word2Vec is actually a two layer neural network to process text for finding the word embedding. \etal~\cite {Deocadez:2017} performed semi-supervised classification techniques for automated classification of FR and NFR in user reviews from the app store. This study deals with two problems: 1) minimize annotate or label the big dataset of user reviews 2) classification of FR and NFR. First problem solved by using semi-supervised self-labeling algorithm. Self-labelling algorithms needs only small amount of dataset to get the comparable results with supervised techniques. For classification problem Naïve Bayes algorithm used. Features are obtained by applying standard text mining technique and additional attribute embellishment. For text mining technique following features used:  Inverse Document Frequency (IDF) Transform, Term Frequency (TF) Transform, Lowercase transformation, Minimum term frequency, Stemmer, Number of words. The second stage involved removing numbers, 2- letter words and other symbolic characters.\etal~\cite{Kurtanovic:2017}performed automated analysis of number of software requirements and performed classification on FR, NFR and its different categories using supervised machine learning algorithm(support vector machine). Additional dataset of user comments from Amazon used to overcome the data imbalance problem in NFR. For feature extraction text preprocessing techniques used such as removal of punctuations, removal of stop words, and lemmatization. 

\cite{Deocadez:2017}, \cite{Kurtanovic:2017}, \cite{Guzman:2017},
\cite{Abad:2017}, \cite{Dekhtyar:2017}, \cite{Rashwan:2012}, \cite{Lu:2017},
\cite{Hayes:2014}, \cite{Williams:2017}, \cite{Garzoli:2013},
\cite{Casamayor:2010}, \cite{Wang:2016}, \cite{Hussain:2012}, \cite{Jiang:2014},
\cite{Jha:2017}, \cite{Pinquie:2015}

\subsubsection{Security Requirements} 

Due to the orthogonal character of their impact on a system, \emph{security}
requirements are notoriously difficult to identify, objectify and
quantify~\cite{}. Also during requirement specification, it very often
happens that security requirements are masked by functional requirements
(but can be deduced from the context of the domain the system operates
in)~\cite{Riaz:2014}. Because of this, it often happens in practice that
security requirements are only marginally tackled during system
construction~\cite{}, paving the way to potentially catastrophic consequences.
Machine learning can be of use here by aiding in the identification of segments
of text that describe security requirements. This is a \emph{classification}
problem: given a text, identify which parts of it correspond to which
type of security issues.\\

Jindal\etal~\cite{Jildal:2016} automatically learn decision trees that can be
used to classify security requirements as \emph{authentication}, \emph{access
control}, \emph{encryption} or \emph{data integrity}. The \emph{features} used
are relevant terms found in the text. Such relevant terms are obtained by the
following sequence of actions: 1) removing stop words from the text; 2) stemming
the remaining words; and 3) ranking the stemmed words by their \emph{info-gain}
measure.

Riaz and her colleagues~\cite{Riaz:2014} use the k-nearest neighbors algoritm to
classify sentences in requirements documents as \emph{confidentiality},
\emph{integrity}, \emph{authentication}, \emph{availability},
\emph{accountability} or \emph{privacy} requirements. In order to find
adequate sentences and provide context to the classifier, the authors start by
finding a type for each sentence among the possibilities \emph{title}, \emph{list start}, \emph{list
element} or \emph{normal sentence}. For the classification the authors use a
modified version of the Levenshein distance based on the number of word transformations needed to go
from one term in one sentence to a term in another sentence. The classifier is
trained using requirements sentences from the healthcare domain that are
manually classified. A particularity of the approach is that each security
requirement type is associated to a template that helps in translating the
security requirements into functional requirements in order to ease during the
implementation of the final system. 


\cite{Knauss:2011} 

\subsubsection{Contextual Requirements} 


\subsection{Requirements Validation}

\subsubsection{Traceability}

Validation is to guarantee that requirements are reflecting stakeholders' needs,
confirm the quality of the system, consistency, and traceability.
In requirements traceability, the emphasis is on the ability to track the life of
requirements and their established links within other artifacts. However, the
main barrier assures traceability is the needed effort for building and
maintaining the links between those artifacts. That is why many research has
tried to apply machine learning and automated tools for facilitating the
establishment of links\cite{Gervasi:2011}.
Traceability tackled in the research mainly by the use of machine learning
classification and reinforcement learning methods.

Gervasi \etal~\cite{Gervasi:2011} investigate what can be learned from links that are already established. They build classifiers as a mean to develop models of tracing that can then be interpreted by humans to understand how requirement tracing is done in practice. Their purpose is to revise the existing models of hard-coded traceability tools such as VSM. They used a publicly-available dataset of requirements with traceability information, originally based on the CM-1 project by the NASA Metrics Data Program. Their approach has the following steps: applied preprocessing techniques and transferred requirements into a vector of features which from them derived set of classification cases by joining one high-level requirement and one low-level requirement and adding a classification of \textit{link} or \textit{nolink} based on whether that particular pair was a true link in the original dataset, or not. Finally, use the dataset to train and test two different classifiers from the WEKA collection, a Naive Bayesian classifier, and the J48 decision-tree classifier.

Sultanov \etal~\cite{Sultanov:2013} finds traceability candidates from
high-level to low-level requirements by the use of reinforcement
learning. They used textual high and low-level requirements documents as an input and try to
find the candidate traces. Their technique demonstrated statistically
significantly better results than the Information Retrieval technique.






\subsection{Requirements Management}

\subsubsection{Visualization}

Natural language requirement documents can be hard to comprehend and analyze.
Similarly, stakeholders have to review and understand requirements for large and
complex systems.  In these scenarios, basic information visualizations, like
charts and graphs have been used in requirements engineering.
These visualizations are usually applied to improve textual requirements with
summarization that combined large amounts of information into a single
representation for quick consumption by stakeholders\cite{Reddivari:2012}.
Machine learning is of great value in discovering visualized groups of large
numbers of requirements artifacts\levi{I think you mean ``machine learning
is useful in grouping requirements for visualization purposes''}.\levi{which
kind of machine learning taks? Clustering?}\\

ReCVisu (Requirements Clustering Visualization) tool is presented in Reddivari
\etal~\cite{Reddivari:2012} paper. ReCVisu, an exploration tool based on
quantitative visualizations helps requirements engineers understand the nature
of the requirements in a visual form. In ReCVisu, the dependence graph consists
of requirements artifacts as nodes and the textual similarities as edges. The
automatic grouping of requirements into clusters can help in areas such as
uncovering the requirements structure, navigating around the requirements space,
modularizing crosscutting concerns, and understanding requirements interactions
and evolution.\newline

Pinqui \etal~\cite{Pinqui:2015} recognize the enormous volume of requirements as
big data with which companies struggle to make strategic decisions early on.
Therefore, they built a complete visual framework to filter requirements from
stakeholders in a way that architects can make better insightful decisions. They
suggest training a multi-class SVM model from domain-specific (mechanics,
electronics, etc.) dictionaries and handbooks. Overall, the paper proposes a
framework to go from management-oriented requirements to architecture-oriented
requirements in which SVM is only applied in a small part of it. \newline

Software requirements are mostly stated in natural text notations such as user
stories which is making it hard for people to develop an accurate mental image
of the most relevant entities and relationships. Lucassen
\etal~\cite{Lucassen:2016} introduced an automated method for visualizing
requirements at different levels of granularity. Their visualization method from
user stories consists of 1) the generation of an overview which provides a
general context for understanding the dataset:
\begin{itemize}
\item Extract a set of relevant concepts from the user stories and their relationships 
\item Calculate the semantic similarity by using skip-gram implementation word2vec
\item Utilize Ward’s clustering algorithm to group all the concepts according to their similarity 
\item Identify the concept which is most similar to the collection of concepts in a cluster
\item Generate inter-cluster relationships matrix
\item Visualization Drawing
\end{itemize}
2) zooming in and out mechanisms and 3) filtering techniques to reduce the
complexity of the data presentation. Possible anticipated applications of this
visualization are: discovering missing relationships between clusters that may
result in further user stories, teaching system functionality by exploring
simplified, manageable chunks, and analyzing expected system changes after
introducing new sets of user stories.

\subsubsection{Structuring Documents} 

Requirements of the system are usually presented in natural language documents. These documents require to be properly structured for a better overall understanding of the requirements. For this purpose, the document should be organized with independent sections which each one contains conceptually connected requirements\cite{Ferrari:2013}. Moreover, technical review is a usual way to guarantee the quality in natural language specifications. However, extensive and comprehensive specifications make it problematic for reviewers to find defects, especially consistency or completeness ones. Therefore, use of machine learning algorithms can support reviewers with their work by automatically classifying and clustering the information that is spread over many sections of many documents \cite{Ott:2013}.


\cite{Ferrari:2013}

\cite{Rauf:2011}

\cite{Ott:2013}
 
\section{Discussion}
\label{sec:discussion}

Our survey work implicitly points to a number of trends that we will
concretize and summarize in this section. Note that while the pointers we
provide here are informed by the literature review we conducted, this survey is
not fully systematic (as described in section~\ref{sec:threats_to_validity}) which
means our conclusions may be revised and/or extended by future surveys of the
domain.

Table \tab{tab:summary} summarizes our findings. It provides partial answers to
\textbf{RQ2} (``What types of learning methods are used when \ML is applied to
\RE?'') and \textbf{RQ3} (``Which are the \RE problems that are currently using
\ML methods?'') in columns \emph{ML Task} and  \emph{themes}, respectively.
The table also provides partial responses to \textbf{RQ1} (``What is the current
state of the practice in \ML${\&}$\RE?") and \textbf{RQ4} (``Is using \ML
methods improving \RE?''). The answer to \textbf{RQ1} seems to be ``at its
beginning'', given the prevalent lack of comparison with the state of the art
as can be observed in \tab{tab:summary}. The answer to \textbf{RQ4} is
``unknown'', given that most of the studies read by us were initial proposals
with little academic or industrial validation in real software engineering tools or projects.

Note that \tab{tab:summary} provides additional information on which types of
algorithms are used for each kind of theme, as well as datasets used for
learning and which are available online.

It is obvious from our survey that \NLP techniques are heavily used
throughout a majority of the research tacking the application of \ML to
\RE. This is not surprising and even intuitive. \RE is the area of software
engineering where natural language is employed more ubiquitously, as \RE
techniques and tools play the role of interface between stakeholders such as
clients, certification entities, architects or developers. Although many
attempts have been done to bring formality to requirements
engineering~\cite{Teufl17,LucioRCA16}, the \emph{de facto} language between
technical and non-technical stakeholders for real-world projects continues being
natural language, and in particular English.
The IBM Rational \DOORS family~\cite{doors} of tools is an example
of a natural-language based tool for requirement engineering that has become the
reference in many domains. In the techniques we have observed, \NLP is heavily
used for the preprocessing stages of natural language in order to bring the data to a format
that can be consumed by a learning algorithm (see
section~\ref{sec:preprocessing}).
 
The authors of the articles we have processed in our survey of point
to the idea that \ML can potentially bring about enormous benefits in terms of
processing and taking decisions based on  large amounts of imprecise and
ambiguous data\levi{some citations here}. In the real-world of software
engineering, parsing and summarizing requirements is a very time-consuming
activity. Also, decisions taken by technical stakeholders are often based on imprecise,
incomplete and noisy information and are supported by rules-of-thumb, experience
and intuition. \ML is by nature built to handle and cope with such challenges
-- it based on data and it's main purpose is exactly to build model of patterns
that humans associate to rules-of-thumb, experience or intuitions. Additionally,
\ML methods often provide a precise degree of certainty regarding the
correctness of decisions taking during a software engineering project. Such
measures, although valid regarding the quality of the learning process, allow
assessing the risk associated to certain steps in the course of a project.

In the sequence of the previous paragraph, a large set of datasets
on \RE are available online. We have identified a few of such datasets in
\tab{summary}. This fact is a cornerstone for the domain, as most \ML
algorithms existing nowadays are vey data-intensive. One of the authors of this
survey has recently written a similar article on the application of \ML to
formal verification~\cite{AmLuBi:2018}, for which the datasets available
to learn from are typically very small and almost never made public. The authors
of the article recognize that such scarceness of data is partly due to the niche
nature of the domain of formal verification, where the datasets are mostly in
the form of mathematical proofs. Nonetheless, and in spite of the large body of
work regarding the application of \ML to formal verification, such scarceness of
data poses a problem not only to the automated learning, but also to the
scientific validation of such proposals. This is not the case in \RE, where many
datasets are publicly available on which both learning and validation can be
done.

The majority of the articles we found on the topic of \ML and \RE have to do
with either the \emph{elicitation} or the \emph{analysis} phases of \RE. These
findings are compatible with the idea that parsing requirements texts and
classifying the information that is contained in them is strenuous for humans
and thus it is desirable that such tasks are as automatic as possible. The
\emph{validation} and \emph{management} phases in \RE also imply tasks that can
be automated as we have shown through our survey, but the state of the art in
the domain seems to imply that the first two phases have priority for
researchers and practicioners.

Also, we have observed through our readings that while \emph{classification} is
the most used \ML task, \emph{clustering} also plays an important role in the
domain of \ML applied to \RE. This contradicts the results
in~\cite{AmLuBi:2018}, where \emph{clustering} has almost no expression in work
that applied \ML to formal verification. We believe this provides support to the
thesis that \ML is particularly appropriate to \RE -- given \emph{clustering} is
especially useful when mining non-formal data such as free-form text.

\newcommand\nocell[1]{\multicolumn{#1}{c|}{}}
\begin{table*}%
	\begin{scriptsize}
\begin{center}
	\begin{tabular}{|c|l|l|l|l|l|}
		\cline{2-6}
		\nocell{1} & \textbf{Themes} & \textbf{Contributions} & \textbf{ML Task} &
		\textbf{\ML Model Types} & \textbf{Datasets
		Used}\\
		\cline{2-6}
		\cline{1-6} 
		\multirow{3}{*}{\rotatebox[origin=c]{90}{\textbf{E}}}
			& External  & \cite{Guzman:2017}($\circ$)  \cite{Williams:2017}(+)
			\cite{Jiang:2014}(+) & Classification & (Multinomial) Na\"ive Bayes & Online
			reviews for KIS 2011 (from Amazon) \\
			&& \cite{Douglas:S2008}($\circ$) \cite{Jha:2017}(+)
			\cite{Castro-Herrera:2009}($\circ$) & Clustering & Support Vector Machines &
			Skiweb data \\
			\cdashline{2-6}[.4pt/1pt]
		\hline
		\multirow{3}{*}{\rotatebox[origin=c]{90}{\textbf{S}}} 
			& Non-Functional & ~\cite{Slankas:2013}($\circ$) ~\cite{Cleland-Huang2007}($\circ$)& Classification & k-Nearest Neighbors & Open Source PROMISE Datase\footnote{http://promise.site.uottawa.ca/SERepository/datasets-page.html}\\
			\cdashline{2-6}[.4pt/1pt]
			& Functional & ~\cite{7949577}($\circ$) & Classification & Bi-Directional
			Long Short-Term Memory &\\
			&&&&Conditional Random Field Network & - \\
			\cdashline{2-6}[.4pt/1pt]
			& Functional \& & \cite{Lu:2017}($\circ$) \cite{Deocadez:2017}($\circ$) \cite
			{Kurtanovic:2017}($\circ$) & Classification  &  Bagging,  Na\"ive Bayes, SVM & Open Source PROMISE Dataset \\
			& Non-Functional & \cite{Abad:2017}($\circ$) \cite {Garzoli:2013}($\circ$)
			~\cite{Wieloch:2013}($\circ$)  & & & app-store reviews \\
			\cdashline{2-6}[.4pt/1pt]
			& Prioritization & \cite{S.Dhingra}($\circ$) \cite{PAvesani}(+)
			\cite{7320432}($\circ$) & Classification & Case Based Ranking \ J.48 DT &  -
			\\
			\cdashline{2-6}[.4pt/1pt] 
			& Security & \cite{Jildal:2016}($\circ$) \cite{Riaz:2014}($\circ$) &
			Classification & Decision-Tree &\\&&&& k-Nearest Neighbors & - \\
			\cdashline{2-6}[.4pt/1pt]
		\hline
		\multirow{3}{*}{\rotatebox[origin=c]{90}{\textbf{V}}} 
			& Traceability & ~\cite{Gervasi:2011}($\circ$) ~\cite{Sultanov:2013}($\circ$) & Classification &  Na\"ive Bayes / J48 Decision-Tree & Open Source CM-1 NASA project\footnote{http://promise.site.uottawa.ca/SERepository/datasets/cm1.arff}\\
 		    & & & Reinforcement Learning & & Open Source Pine
 		    Dataset\footnote{http://www.washington.edu/pine/} \\
			\cdashline{2-6}[.4pt/1pt]
		\hline
		\multirow{3}{*}{\rotatebox[origin=c]{90}{\textbf{M}}}
			& Visualization & ~\cite{Reddivari:2012}($\circ$) ~\cite{Pinqui:2015}($\circ$) ~\cite{Lucassen:2016}($\circ$) & Classification & Support Vector Machines & \\
			& & & Clustering & Ward's method & - \\

			\cdashline{2-6}[.4pt/1pt]
			& Structuring & ~\cite{Duan:2007}($\circ$)
			~\cite{Winkler:2016}($\circ$)~\cite{Rauf:2011}($\circ$) & Classification & Multinomial  Na\"ive Bayes & International Union of Railways\\
			& & ~\cite{Ferrari:2013}($\circ$) ~\cite{Ott:2013}($\circ$) & Clustering &
			Support Vector Machines & (EIRENE
			Functional Requirements Specification\footnote{https://www.transportstyrelsen.se/globalassets/global/jarnvag/vagledning/godkannande/eirene($\_$)frsv7.pdf})\\
			& & & &  Convolutional Neural Networks & Mercedes-Benz car development\\
			&&&& Sliding Head-Tail Component Clustering & \\ 
			&&&& Hierarchical Clustering &\\
			\cdashline{2-6}[.4pt/1pt]
		\hline
	\end{tabular}
\end{center}
	\center{Legend: (+) improves the state of the art; (-) comparable to or worse
	than state of the art; ($\circ$) no information on how the approach
	relates to the state of the art}
	\end{scriptsize}
	\caption{Contributions and \ML tasks related to each theme within each \RE
	approach.}
	\label{tab:summary}
\end{table*}


\levi{internal vs external} 
\section{Threats to Validity}
\label{sec:threats_to_validity}

The validity of the study might be affected by the coverage of the search results, bias on study selection, and inaccuracy of data extraction.
\paragraph{Study Coverage.}
Some relevant studies could be missing in our study due to inadequate search strings or missing databases.
To cope with this threat, the data preparation was based on a systematic method.
\vspace{-5pt}
\paragraph{Study Selection Bias.}
Study assessment might be biased by researchers.
To mitigate for this threat, a set of include and exclude criteria was predefined and researchers assessed the title and abstract of the papers based on them to steer the assessment.
\vspace{-5pt}
\paragraph{Inaccuracy of Data Extraction.}
Also the data extraction process might be biased by researchers.
To mitigate for this threat, the selection of data items was strictly driven by the research questions.
Moreover, assignments were marked by the researchers depending on their confidence level. Low-confidence assignments were discussed between the authors until a consensus was reached.   
\section{Conclusion}

Through our bird's eye view of \ML applied to \RE we have observed that in the
past couple of decades a good amount of research has been done on how to bring
these two worlds together. The stakes are high: while requirements engineering
is currently a domain under intensive research, attempts to address its
challenges academically have translated into few results in practice. Free-form
text-based tools with light-weight structuring capabilities such as DOORS are
now the norm in practice. Requirements elicitation, analysis, validation and
management keep on relying on human expertise and talent. While academics often insist that
better formalization brings advantages, the languages in which requirements are
formalized do not match the need that stakeholders in the \RE process
(technical and non-technical) need to communicate through artifacts that are
intelligible to all.

While not overstating the potential of \ML, which is has its own challenges to
overcome such as coarseness of the learned models, overfitting or hungriness
for data, we have provided in this article indications that \ML might become a cornerstone in \RE. For now, it
seems like the domain is now undergoing a pre-scientific phase: the studies we
have analyzed seldom compare themselves with the state-of-the-art (see
\tab{summary}). This suggests that the current body of research is composed of
new ideas, which have not yet been validated to its full extent by the
scientific or industrial communities. We thus call for a more extensive survey to validate
the preliminary conclusions we present in this work.
 
  
\bibliographystyle{abbrv} 
\bibliography{./ml_req}  

\end{document}
